\chapter{Summary}\label{summary}
This dissertation is set to construct a Hamiltonian formulation for the generalized harmonic interpretation of general relativity. As an en route investigation, this dissertation defines a missing notion of the 3 + 1 decomposition, the covariant time derivative $D_{t}$. As an analogy to covariant spatial derivative, the operator $D_{t}$ defines a covariant version of the ordinary time derivative $\partial_{t}$. Since the expression of $D_{t}$ is transformation dependent, in this dissertation it is defined under a general class of coordinate transformations, the foliation preserving transformations, which consist of time reparameterization and time dependent spatial diffeomorphisms. General covariance was one of the postulates Einstein posted when he was constructing his theory of relativity and has always been a major part of the elegance credited to it. Therefore we take advantage of the constructed covariant time derivative to recover general covariance of the moving puncture gauge condition in the BSSN formulation and the damped-wave gauge condition in the generalized harmonic equations. 

Later in this dissertation, a Hamiltonian density is derived directly from the predefined action of the general harmonic formulation. It is then recognized as lacking covariance under our foliation preserving transformations. The covariant time derivatives allow us to extend the Hamiltonian density to achieve a set of manifestly covariant Hamilton's equations. This system of covariant evolution equations is later proved to be symmetric hyperbolic by following a series of well-established prescriptions. Hence we conclude that the Hamiltonian approach preserves the well-posedness of the generalized harmonic formulation. 

The symmetric hyperbolicity of our newly developed Hamiltonian system suggests it could be used in numerical applications. Although no numerical experiments are included in the scope of this work, we hope a numerical simulation will be built upon the Hamilton's equations listed in this dissertation in order to test its practical performance. With the covariant time derivatives and gauge conditions defined in this dissertation, we hope to improve the effectiveness and efficiency of existing numerical relativity formulations when multiple coordinate systems are required. Furthermore, the Hamiltonian formulation of generalized harmonic formalism presented in this dissertation introduces an alternate algorithm for simulating the Einstein field equations. Meanwhile we have also presented a different perspective toward numerical relativity and hope that new insights and theories will be sparked by our work. 