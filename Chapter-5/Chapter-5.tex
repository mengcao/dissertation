\chapter{Summary}\label{summary}
This paper is set to construct a Hamiltonian formulation for generalized harmonic interpretation of general relativity. As an en route investigation, this paper defines a missing notion of the 3 + 1 decomposition, the covariant time derivative $D_{t}$. As an analogy to covariant spatial derivative, the operator $D_{t}$ defines a covariant version of the ordinary time derivative $\partial_{t}$. Since the expression of $D_{t}$ is transformation dependent, in this paper it is defined under a general class of coordinate transformation, the foliation preserving transformation, which consists of a time reparameterization and a time dependent spatial diffeomorphism. General covariance was one of the postulates Einstein posted when he was constructing his theory of relativity and has always been a major part of the elegance credited to it. Therefore we take advantage of the constructed covariant time derivative to recover general covariance of moving puncture gauge condition in BSSN formulation and damped-wave gauge condition in generalized harmonic equations. 

Later in this paper, a Hamiltonian density is derived directly from the predefined action of general harmonic formulation. It is then recognized as not covariant under our foliation preserving transformation. Understanding of the covariant time derivatives under foliation preserving transformation provides us with insights about how to extend the Hamiltonian density to achieve a set of manifestly covariant Hamilton's equations. This system of covariant evolution equations is later proved to be symmetric hyperbolic by following a series of well-established prescriptions. Hence we conclude that the Hamiltonian approach to generalized harmonic formulation preserves its well-posedness. 

The symmetric hyperbolicity of our newly developed Hamiltonian system guarantees its empiricalness in numerical applications. Although no numerical experiments are included in the scope of this work, we urge a numerical simulation to be build upon the Hamilton's equations listed in this paper in order to test its practical performance. With the covariant time derivatives and gauge conditions defined in this paper, we hope to improve the robustness, effectiveness and efficiency of existing numerical relativity formulations when multiple coordinate systems are required. Furthermore, the Hamiltonian formulation of generalized harmonic formalism presented in this paper introduces an alternate algorithm for simulating Einstein field equations. Meanwhile we'd also like to present a different perspective toward numerical relativity and hope new insights and theories can be sparkled by our work. 