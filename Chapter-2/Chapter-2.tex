\chapter{Numerical Relativity}\label{nr}
As mentioned before, the rise of numerical relativity answered the need for a wide class of general solutions to Einstein's equations for various physical phenomena, such as perturbed black holes, coalescence of black holes and neutron stars, which general relativity must explain. As Wald pointed out,``If a corresponding large class of solutions of Einstein's equation failed to exist, we would be forced to reject general relativity as a correct theory of nature''\cite{Wald:GRbook}. Inevitably, in order to employ computers to tackle dynamical evolution problems, we have to separate the time and space dimensions Einstein united in his theory, i.e. convert Einstein's equations into an initial value problem. The 3 + 1 decomposition scheme to split time and space was initially proposed by Arnowitt, Deser and Misner\cite{ADM:Witten} and a detailed discussion of it is presented in Sec.~\ref{3+1}. Mainstream numerical formulations of general relativity are presented in Sec.~\ref{numericalformulation} and their limitations are discussed. 
\section{3 + 1 Decomposition}\label{3+1}
For initial value problems of ordinary particle mechanics, the evolution of a system is dependent on the initial conditions and equations of motion for the system. It is easy to feed computers with initial value data and integrate the partial differential equations of motion along the time axis numerically to obtain the dynamic evolution of the system. However, since Einstein united time and space in his theory of gravity, initial value data is not native in Einstein's equations. Therefore, an important prerequisite for developing any numerical formulation for the Einstein field equations is to decompose the four dimensional spacetime into a one-parameter family of nonintersecting spacelike hypersurfaces, namely a foliation, ``one slice for each instant of time''. The operation of constructing a rigid structure of such slices is referred as ``3 + 1 decomposition'' and it is demonstrated in Fig.~\ref{3+1diagram} (a thorough discussion of 3 + 1 decomposition can be found in Ref.~\cite{Gourgoulhon:2007ue}). 

\begin{figure}[hbtp]
\centering
\includegraphics{decomposition-1}
\caption{A spacetime diagram illustrating the 3 + 1 decomposition. $t^{\mu}$ is the flow of time vector; $n^{\mu}$ is the normal vector on the lower slice; $g_{ab}$ is the spatial metric for the lower slice; $\alpha$ is regarded as the lapse function while $\beta^{a}$ is known as the shift vector. }
\end{figure}\label{3+1diagram}

In order to  construct a rigid structure, several physical quantities need to be specified. First and foremost is the three dimensional spatial metric $g_{ab}$ on each slice, which defines the proper distance between two points residing on the hypersurface. Second of all, the proper distance between each slice is specified by the lapse function $\alpha$. Although it is not necessary to keep the coordinate system on successive slices related to each other, due to practical considerations, a shift vector $\beta^{a}$ is defined to weld the coordinate systems between consecutive hypersurfaces. The spacetime metric $^{(4)}g_{\mu\nu}$ is split in terms of the spatial metric $g_{ab}$, lapse function $\alpha$ and shift vector $\beta^{a}$ as follows
\begin{equation}
	{}^{\left(4\right)}g_{\mu\nu} = \left(-\alpha^2 + \beta^{a}\beta_{a}\right)\delta_\mu^0\delta_\nu^0 
	+ 2\beta_{a}\delta_{(\mu}^a\delta_{\nu)}^0 + g_{ab} \delta^a_\mu \delta^b_\nu \ ,
\end{equation}
where $\delta^{\mu}_{\nu}$ is the usual delta tensor. 

The covariant normal to each foliation is denoted as
\begin{equation}\label{normal covector}
n_{\mu} \equiv -\alpha\delta^{0}_{\mu} \ ,
\end{equation}
and its dual is
\begin{equation}\label{normal vector}
n^{\mu} \equiv \left(\delta^{\mu}_{0} - \beta^{c}\delta^{\mu}_{c}\right)/\alpha
\end{equation}
so that we have $n_{\mu}n^{\mu} = -1$. 

There is also the projection operator
\begin{equation}\label{projection 1}
X^{\mu}_{a} \equiv \delta^{\mu}_{a}
\end{equation}
that projects a spacetime covector into a spacelike covector. Its covariant form is
\begin{equation}\label{projection 2}
X^{a}_{\mu} \equiv \delta^{a}_{\mu} + \beta^{a}\delta^{0}_{\mu}
\end{equation}
so that we have $X^{\mu}_{a}X^{b}_{\mu} = \delta^{b}_{a}$ and $X^{\mu}_{a}n_{\mu} = 0$. 

With these definitions, the spacetime metric can be written in terms of the spatial metric, the normal and the projection operator as
\begin{equation}\label{spacetime metric 3 + 1}
^{\left(4\right)}g_{\mu\nu} = g_{ab}X^{a}_{\mu}X^{b}_{\nu} - n_{\mu}n_{\nu}
\end{equation}
Spacetime indices $\mu$, $\nu$, ... are always raised and lowered with the spacetime metric $^{\left(4\right)}g_{\mu\nu}$ and its inverse $^{\left(4\right)}g^{\mu\nu}$, while spatial indices $a$, $b$, ... are always raised and lowered with the spatial metric $g_{ab}$ and its inverse $g^{ab}$. 

The notion of extrinsic curvature in the 3 + 1 context represents the time derivative of the spatial metric for the foliation. It is defined as
\begin{equation}\label{extrinsic}
K_{ab} = - \frac{1}{2\alpha}(\partial_{t} - \mathcal{L}_{\beta})g_{ab} \ ,
\end{equation}
and it is easy to show that $K_{ab}$ is a symmetric tensor, i.e., $K_{ab} = K_{ba}$. Physically, the extrinsic curvature, as an abstract coordinate-independent geometric object, defines the curvature of a three dimensional hypersurface relative to the four dimensional spacetime manifold it is embedded in. Imagine parallel transporting the normal covector $n_{\mu}$ of slice $\Sigma$ from point $p$ to point $q$. This parallel transported covector will fail to coincide with the normal covector $n_{\mu}$ at $q$. This failure reflects the bending of $\Sigma$ in spacetime and the extrinsic curvature directly measures the deviation between the parallel transported covector and $n_{\mu}$ at $q$. 

With the help of the 3 + 1 decomposition, one is able to specify a set of dynamic variables ( $g_{ab}$, $K_{ab}$, etc. ) on one slice and utilize it as the initial condition. Different variations of the Einstein field equations have been developed to integrate the initial data along time and obtain a uniquely determined dynamical evolution of the spacetime geometry. Several important types of  numerical formulations are discussed in Sec.~\ref{numericalformulation}.  
%%%%%%%%%%%%%%%%%%%%%%%%%%%%%%%%%%%%%%%%%%%%%%%%%

%%%%%%%%%%%%%%%%%%%%%%%%%%%%%%%%%%%%%%%%%%%%%%%
\section{Numerical Formulations of General Relativity}\label{numericalformulation}
This section briefly reviews the development of numerical formulations for solving Einstein's equations while detailed review of this topic can be found in Ref.~\cite{lrr-2012-9}. Two widely adopted modern numerical formalisms are discussed in detail. Our motivation for this work is reviewed at the end of this section. 

\subsection{ADM and BSSN formalisms}
The foundation for solving Einstein field equations numerically was laid down by Arnowitt, Deser and Misner via their famous ADM formalism\cite{ADM:Witten} in 1959. In the original ADM paper, a Hamiltonian approach was taken toward rewriting Einstein's equations. The Hilbert-Palatini gravitational action was written in terms the spatial metric $g_{ab}$, its conjugate momentum $\pi^{ab}$, lapse function $\alpha$ and shift vector $\beta^{a}$. The dynamic variables are chosen to be $g_{ab}$ and $\pi^{ab}$. Variation of the action with respect to the spatial metric and its momentum yields evolution equations for $g_{ab}$ and $\pi^{ab}$, while varying the action with respect to the lapse function and shift vector provides constraint equations for the initial conditions of the dynamic variables. These equations, obtained by varying the action, determine how the system evolves with time. 

In the numerical system now commonly referred to as the ADM formalism\cite{Smarr:York,Smarr:1977uf}, the extrinsic curvature $K_{ab}$ replaces $\pi^{ab}$ as one of the dynamic variables and they are closely related as $\pi^{ab} = -\sqrt{g}(K^{ab} - g^{ab}K)$. The evolution equations of $g_{ab}$ and $K_{ab}$ drive the system dynamically. Two sets of constraint equations known as the Hamiltonian and momentum constraints govern the choice of initial data for the spatial metric and extrinsic curvature. These equations together are equivalent to the Einstein field equations. 

Unfortunately, the ADM formalism was later proved to be impractical for the construction of numerical simulations because the system of equations is only weakly hyperbolic. As mentioned in Chap.~\ref{wellposedness}, weak hyperbolicity guarantees the ill-posedness of the system while strong hyperbolicity is a necessary condition for the system to be well-posed in the absence of spatial boundaries. Furthermore, symmetric hyperbolicity is a necessary condition for the system to be well-posed in the presence of spatial boundaries. The notion of well-posedness is discussed in detail in Chap.~\ref{wellposedness}. Essentially well-posedness defines the system's analytical stability under physical perturbations of the initial data. This means that for a ill-posed system, such as the ADM equations, small changes in the initial data might lead to non-proportional changes of the solution and this is the direct reason of ADM formulation's incompetence for numerical application. 

During the years between 1987 to 1999 Baumgarte, Shapiro, Shibata and Nakamura\cite{Shibata:1995we, Baumgarte:1998te} developed a strongly hyperbolic modification of the ADM formalism. This modified system is known as BSSN formalism and is one of the most commonly used numerical formulation for relativity applications. 

Instead of evolving the spatial metric and extrinsic curvature, the BSSN formalism chooses a conformal factor $\phi$, the trace of extrinsic curvature $K \equiv K_{ab}g^{ab}$ and a set of conformal variables, namely the conformal spatial metric ${\tilde g}_{ab} \equiv e^{-4\phi}g_{ab}$, conformal trace-free extrinsic curvature ${\tilde A}_{ab} \equiv e^{-4\phi}(K_{ab} - \frac{1}{3}g_{ab}K)$ and conformal connection functions ${\tilde \Gamma}^{a} \equiv {\tilde g}^{bc}{\tilde \Gamma}^{a}_{~bc}$, as the fundamental variables to evolve along with time. Supplemented with the moving puncture gauge conditions which govern evolution of the lapse function and shift vector, BSSN formalism is widely used in the numerical relativity community to analyze a variety of gravitational problems. The BSSN equations are strongly hyperbolic. 

\subsection{Generalized Harmonic Formulation}
Another well recognized numerical formulation of Einstein's equations is the generalized harmonic(GH) equations\cite{Friedrich:1985, Garfinkle:2001ni, Pretorius:2006tp}. As many numerical algorithms are recognized as stable and efficient when dealing with wave equations, the GH formulation writes Einstein field equations in a form which is similar to curved spacetime wave equations by utilizing harmonic coordinates. The generalized harmonic coordinates used by the GH formulation satisfy the wave equations
\begin{equation}
	\nabla^{\alpha}\nabla_{\alpha}x^{\mu} = H^{\mu} \ ,
\end{equation}
where $H^{\mu}$ is an freely chosen source vector\cite{Friedrich:1985, Friedrich:1996, Garfinkle:2001ni}.

The GH version of Einstein field equations can be expressed in an abstract form as
\begin{equation}\label{ghequations}
	g^{\alpha\beta}\partial_{\alpha}\partial_{\beta}g_{\mu\nu} + \partial_{(\mu}H_{\nu)} = Q_{\mu\nu}(H, g, \partial g) \ , 
\end{equation}
where $Q_{\mu\nu}$ are functions that depend on the source vector, spacetime metric and its first-order derivatives. Equations for the source terms $H^{\mu}$ are evolved along with Eqs.~\ref{ghequations} to solve Einstein's equations numerically. These equations are referred to as gauge conditions. Gauge conditions\cite{Lindblom:2007xw, Lindblom:2009tu} have been developed to combine with Eqs.~(\ref{ghequations}) such that the complete system of equations is symmetric hyperbolic\cite{Lindblom:2005qh}. 

In recent years, the 3 + 1 form of the GH equations\cite{Brown:2011qg} has been developed
to reveal the relationship between the GH, ADM and BSSN formulations. Furthermore, the action principle for the GH formulation\cite{Brown:2010rya} is presented to provide an alternative perspective to understanding this formalism. The foundation laid down by these two pieces of work and the significant role that the Hamiltonian perspective has played in history of physics directly motivate us to develop the Hamiltonian system of the GH equations. 

During our research on the Hamiltonian approach to the GH formalism, we found that despite their supremacy in practical applications, both BSSN and GH formulations have some drawbacks regarding general covariance. This issue is addressed in Chap.~\ref{generalcovariance} before we move on to the Hamiltonian formulation. 

