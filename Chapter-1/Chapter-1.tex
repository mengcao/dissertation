\chapter{Introduction}\label{background}
The purpose of this paper is to develop a Hamiltonian system for generalized harmonic formulation of general relativity. Our motivation of this specific topic is discussed in Sec.~\ref{gr} and Sec.~\ref{ham} by briefly reviewing Einstein's general relativity and the significant role of Hamiltonian formulation in history of physics. Notation and convention followed throughout this paper are presented in Sec.~\ref{notation}. In the end of this chapter, abstracts of each following chapter is listed in Sec.~\ref{abstracts}.
%%%%%%%%%%%%%%%%%%%%%%%%%%%%%%%%%%%%
\section{General Relativity}\label{gr}
During the years between 1911 and 1916, Einstein developed an elegant geometric approach to generalize Newtonian mechanics and his own special relativity. The new theory formulated an innovative interpretation toward gravitation and was named general relativity. Other than most of natural forces, whose existence is represented by the field defined on spacetime, Einstein proposed in his generalized theory of relativity that gravitation is inherent in spacetime itself and it is the direct result of the curvature of spacetime\cite{carroll2003spacetime}.

When his first paper on this field of study, {\it The Foundation of the General Theory of Relativity}, was published in 1916, it was regarded as a heterodox theory packed with controversial and bold predictions that challenged people's knowledge. Its seemingly incompetence in applications was another obstacle for its adoption by the majority. In 1919, Sir Arthur Stanley Eddington, an astrophysicist from British Royal Society, observed and measured the bending angle of light by the sun to a very small vicinity of Einstein's estimate, which served as the first experimental evidence supported general relativity. Two of his other predictions, gravitational red shift and the precession of the perihelion of Mercury around the sun, were also confirmed by experiments consecutively. From then on, various consequences of general relativity, from the big bang, expanding universe to black holes, were accepted by the mainstream physics society and hence inspired huge amount of young people's interests in physics. After being studied and elaborated for several decades, now it is the standard description of gravitation in modern physics. 

Lying in the core of general relativity is a set of 10 partial differential equations, namely Einstein field equations or Einstein's equations, which governs properties of our spacetime geometry. Although there are several famous analytical solution of Einstein's equations, such as the Schwarzschild solution, it is unarguably a challenge to solve those equations manually for a wide variety of physical phenomena for which general relativity must account. Due to the prosperity of computer science, a new branch of general relativity, numerical relativity, emerged above the horizon. It employs numerical methods and algorithms, leverages power of high performance computers to solve Einstein's equations and hence to study black holes, gravitational waves, neutron stars and other phenomena governed by general relativity.

Numerical simulation also plays a significant role in experimental general relativity. Nearly 100 years ago, Sir Eddington used optical telescope and photographic plates to observe and measure light bent by the sun, thanks to the blooming technology, now we have multiple cutting-edge ground-based facilities stretching in miles listening to gravitational waves, such as LIGO, Virgo and GEO. Moreover, LISA/eLISA, the first space-based gravitational wave telescope, is under construction and is expected to reach a new level of accuracy for detecting gravitational waves. Waveform signals from these sophisticated gravitational wave detectors can be compared to numerical results to gain a better understanding of the sources of gravitational radiation. On the other hand, numerical simulation can also produce artificial templates of gravitational waveforms. With the help of these templates, noise signals can be filtered out from experimental datas picked up by the detectors and hence to magnify the probability of detection. A detailed discussion of numerical relativity is presented in Chap.~\ref{nr}. 
%%%%%%%%%%%%%%%%%%%%%%%%%%%%%%%%%%%%%%%%%%%%%
\section{Hamiltonian Formulation}\label{ham}
Our study toward solving Einstein's equations takes the Hamiltonian viewpoint, rather than Lagrangian approach. Lagrangian mechanics and Hamiltonian mechanics are two mathematically equivalent branches of classical mechanics. In Lagrangian formulation a system with $n$ degrees of freedom is described by $n$ second order partial differential equations of motion with $2n$ initial conditions. The $2n$ initial conditions are usually specified by values of $n$ generalized coordinates $q_{i}$ and $n$ of $q_{i}$'s time derivative, ${\dot q}_{i}$ at time $t_{0}$. From Hamiltonian perspective, the same system with $n$ degrees of freedom is interpreted in $2n$ first order partial differential equations of motion of $2n$ independent variables and their initial values at time $t_{0}$ serve as $2n$ initial conditions. Those $2n$ independent variables are regarded as canonical variables and are usually constructed by $n$ generalized coordinates $q_{i}$ and their conjugate momentums $p_{i}$.

Comparing to Lagrangian procedure, Hamiltonian method doesn't particularly provide a superior solution to mechanics problems. Mathematically, the transition from Lagrangian to Hamiltonian formulation is merely an operation of changing the set of variables $(q_{i}, {\dot q}_{i}, t)$ to $(q_{i}, p_{i}, t)$, namely Legendre transformation. Furthermore, a system's equations of motion from Hamiltonian perspective are practically the same partial differential equations provided by Lagrangian formulation. The power and advantages of Hamiltonian formulation lies in providing a framework for theoretical extensions in many area of physics. With a deeper insight into mechanics, Hamiltonian formulation treats coordinates and momenta equally as independent variables, which entitles physicists to choose any arbitrary yet appropriate physical quantities as coordinates and momenta. A broader selection of dynamic variables leads to a more abstract way of presenting applications to mechanical problems, which plays an essential role in constructing more modern theories. It is Hamiltonian formulation who served as an embarking point from classical mechanics to both statistical mechanics and quantum theory in the past century and it is why it still attracts attentions from the physics society\cite{goldstein}. More details of generalizing Hamiltonian formulation to a field theory as general relativity will be discussed in Chap.~\ref{introduction}. 
%%%%%%%%%%%%%%%%%%%%%%%%%%%%%%%%%%%%%
\section{Notation and Convention}\label{notation}
Inevitably, we need to work with tensors while studying general relativity. For tensor representation, we use Greek letter indices $\mu$, $\nu$, ... to denote spacetime indices from 0 to 3 and Latin letters $a$, $b$, ... for spatial indices from 1 to 3. In places where ambiguity might arise, we use a prefix $^{(4)}$ to denote a spacetime tensor and hence to distinguish it from its spatial counterpart, as in $^{(4)}g_{\mu\nu}$ or $^{(4)}R_{\mu\nu}$. In addition, we employ geometrized  units system throughout this paper, in which $G = c = 1$.

Ordinary partial derivative is denoted as $
\partial_{\mu} \equiv \frac{\partial}{\partial x^{\mu}}$. $\nabla_{\mu}$ stands for the spacetime covariant derivative while $D_{a}$ is the spatial covariant derivative. And $D_{t}$ represents the covariant time derivative which will be discussed later in this paper. Lie derivative operator is denoted as $\mathcal{L}$. 

We follow the convention used by Minser, Thorne and Wheeler\cite{Misner:1974qy}, namely the ``Landau-Lifshitz Spacelike Convertion'' (LLSC). Specifically, signature of spacetime metric $g_{\mu\nu}$ is ( --, +, +, + ), Riemann tensor is defined as 
\begin{equation}
	R^{\alpha}_{~\beta \mu \nu} \equiv 
	\partial_{\mu}\Gamma^{\alpha}_{~\beta \nu} - 
	\partial_{\nu}\Gamma^{\alpha}_{~\beta \mu} + \Gamma^{\alpha}_{~\sigma \mu}\Gamma^{\sigma}_{~\beta \nu} - \Gamma^{\alpha}_{~\sigma \nu}\Gamma^{\sigma}_{~\beta\mu} \ , 
\end{equation}
while the Ricci tensor 
\begin{equation}
	R_{\mu\nu} \equiv R^{\sigma}_{~\mu\sigma\nu} \ . 
\end{equation}
Finally, the Einstein field equations are stated as 
\begin{equation}
	G_{\mu\nu} \equiv R_{\mu\nu} - \frac{1}{2} R g_{\mu\nu} = 8\pi T_{\mu\nu} 
\end{equation}
where $R$ is the Ricci scalar $R^{\mu}_{~\mu}$ and $T_{\mu\nu}$ is the stree-energy tensor. 

\section{Chapter Abstracts}\label{abstracts}
The structure of the rest of this paper is as following. Chap.~\ref{nr} reviews the development of numerical relativity; presents the foundation of solving Einstein's equations numerically: the $3 + 1$ decomposition; discusses several mainstream numerical formulations and reveals their limitations. Chap.~\ref{generalcovariance} studies an important class of coordinate transformation in the context of $3 + 1$ decomposition. General covariance of different numerical formalisms under this type of coordinate transformation is achieved by investigating transformation rules of various field variables and introducing the notion of covariant time derivatives. A Hamiltonian system of generalized harmonic formulation is developed in Chap.~\ref{hamiltonian}. With the knowledge about covariant time derivatives obtained in Chap.~\ref{generalcovariance}, the Hamiltonian system is extended to reach its general covariance. Such a system is further proved to be well-posed, i.e., symmetric hyperbolic, at the end of that chapter. Chap.~\ref{summary} reviews this paper and gives suggestions about possible future development of this project. 
