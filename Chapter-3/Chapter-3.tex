\chapter{Numerical Relativity}\label{nr}
As mentioned before, the rise of numerical relativity answered for the desire of a wide class of general solution to Einstein's equations for various physical phenomena, such as perturbed black holes, coalescence of black holes and neutron stars, which general relativity must explain. As Wald pointed out,``If a corresponding large class of solutions of Einstein's equation failed to exist, we would be forced to reject general relativity as a correct theory of nature''\cite{Wald}. Inevitably, in order to employ computers to tackle dynamical evolution problems, we have to separate time and space dimensions Einstein united in his theory, i.e. convert Einstein's equations into an initial value problem. The 3 + 1 decomposition scheme to split time and space was initially proposed by Arnowitt, Deser and Misner\cite{ADM:Witten} and a detailed discussion of it is presented in Sec.~\ref{3+1}. Another important aspect of numerical formulation, stability and accuracy, is addressed in Sec.~\ref{wellposedness} in the context of well-posedness. Main stream numerical formulations of general relativity are presented in Sec.~\ref{numericalformulation} and their limitations are discussed. 
%\section{Initial Value Problem}\label{initial}
\section{3 + 1 Decomposition}\label{3+1}
For initial value problems of ordinary particle mechanics, the evolution of a system is dependent on the initial conditions and equations of motion for the system. It is easy to feed computers with initial value data and integrate the partial differential equations of motion along time axis numerically to obtain dynamic evolution of the system. During the past several decades, various types of numerical algorithms are developed to solve initial value problems and they are proved to be successful. However, since Einstein united time and space dimensions in his theory of gravity, initial value data is not native in Einstein's equations. Therefore, an important prerequisite for developing any numerical formulation is to decompose the four dimensional spacetime into a one-parameter family of nonintersecting spacelike hypersurfaces, namely foliations, ``one for each instant of time''. The operation of constructing a rigid structure of such series of foliations is referred as ``3 + 1 decomposition''. 

In order to  construct a rigid structure of the decomposition, several physical quantities need to be specified. First and foremost is the three dimensional spatial metric $g_{ab}$ on each foliation, which defines the proper distance between two points reside on the foliation. Second of all, the proper distance between each foliation is specified by the lapse function $\alpha$. Although it is not necessary to keep coordinate system on each foliation related to each other, due to practical consideration, a shift vector $\beta^{a}$ is defined to weld coordinate systems between each two consecutive hypersurfaces. The spacetime metric $^{(4)}g_{\mu\nu}$ is split in terms of spatial metric $g_{ab}$, lapse function $\alpha$, shift vector $\beta^{a}$ as following
\begin{equation*}
	{}^{\left(4\right)}g_{\mu\nu} = \left(-\alpha^2 + \beta^{a}\beta_{a}\right)\delta_\mu^0\delta_\nu^0 
	+ 2\beta_{a}\delta_{(\mu}^a\delta_{\nu)}^0 + g_{ab} \delta^a_\mu \delta^b_\nu \ ,
\end{equation*}
where $\delta^{\mu}_{\nu}$ is regular delta tensor. 

The covariant normal to each foliation is denoted as
\begin{equation}\label{normal covector}
n_{\mu} = -\alpha\delta^{0}_{\mu}
\end{equation}
 And its dual is
\begin{equation}\label{normal vector}
n^{\mu} = \left(\delta^{\mu}_{0} - \beta^{c}\delta^{\mu}_{c}\right)/\alpha
\end{equation}
so that we have $n_{\mu}n^{\mu} = -1$. 

There is also the projection operator
\begin{equation}\label{projection 1}
X^{\mu}_{a} = \delta^{\mu}_{a}
\end{equation}
that projects a spacetime covector into a spacelike covector. Its covariant form is
\begin{equation}\label{projection 2}
X^{a}_{\mu} = \delta^{a}_{\mu} + \beta^{a}\delta^{0}_{\mu}
\end{equation}
so that we have $X^{\mu}_{a}X^{b}_{\mu} = \delta^{b}_{a}$ and $X^{\mu}_{a}n_{\mu} = 0$. 

With these definitions, the spacetime metric can be written in terms of the spatial metric, the normal and the projection operator as
\begin{equation}\label{spacetime metric 3 + 1}
^{\left(4\right)}g_{\mu\nu} = g_{ab}X^{a}_{\mu}X^{b}_{\nu} - n_{\mu}n_{\nu}
\end{equation}
Spacetime indices $\mu$, $\nu$, ... are always raised and lowered with the spacetime metric $^{\left(4\right)}g_{\mu\nu}$ and its inverse $^{\left(4\right)}g^{\mu\nu}$, while spatial indices $a$, $b$, ... are always raised and lowered with the spatial metric $g_{ab}$ and its inverse $g^{ab}$. 

The notion of extrinsic curvature in 3 + 1 decomposition context represents a well-defined time derivative of the spatial metric on the spacelike hypersurface. It is defined as
\begin{equation}\label{extrinsic}
K_{ab} = - \frac{1}{2\alpha}(\partial_{t} - \mathcal{L}_{\beta})g_{ab} \ ,
\end{equation}
and it is easy to show that $K_{ab}$ is a symmetric tensor, i.e., $K_{ab} = K_{ba}$. Physically, the extrinsic curvature, as an abstract coordinate-independent geometric object, defines curvature of a three dimensional foliation relative to the four dimensional spacetime manifold it is embedded in. Imagine parallel transporting the normal covector $n_{a}$ of foliation $\Sigma$ from point $p$ to point $q$. This parallel transported covector will fail to coincide with the normal covector $n_{a}$ on $q$. This failure reflects the bending of foliation $\Sigma$ in spacetime and extrinsic curvature directly measures the deviation between the parallel transported covector and $n_{a}$ at $q$. 

With the help of 3 + 1 decomposition, one is able to specify a set of dynamic variables ( $g_{ab}$, $K_{ab}$, etc. ) on one foliation and utilize it as the initial condition. Different variations of Einstein field equations were developed by various classes of numerical relativity formulations to integrate the initial condition along time and obtain a uniquely determined dynamic evolution of the spacetime geometry in interest. Several of the mainstream numerical formulations are discussed in Sec.~\ref{numericalformulation}, but before that we first need to talk about which properties a proper initial value numerical formalism should satisfy and the most important one is well-posedness.  
%%%%%%%%%%%%%%%%%%%%%%%%%%%%%%%%%%%%%%%%%%%%%%%%%
\section{Well-posedness}\label{wellposedness}
The notion of ``well-posed'' is originated from Jacques Hadamard, a French mathematician. He argues that mathematical models of physical phenomena should have the properties that a unique solution exists and the solution's behavior changes continuously with the initial conditions. Wald gave his own yet similar definition of well-posedness in his book General Relativity. Wald's definition states that in order for a numerical formulation to be well-posed, firstly small changes in initial data should produce only correspondingly small changes in the solution over any fixed compact region of spacetime; and secondly changes of initial data in a region should only be responsible for changes in the solution inside the causal future of this region. The first requirement is necessary since it is impossible to measure initial condition in infinite accuracy. The second property guarantees no signal could propagate faster than the speed of light and keep the numerical formulation under framework of relativity theory. 

The following section briefly reviews the development of numerical formulations for solving Einstein's equations. The generalized harmonic formalism for general relativity is discussed in detail and its limitations and restrictions are revealed. 
\section{Numerical Formulation of General Relativity}\label{numericalformulation}