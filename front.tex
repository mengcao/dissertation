%% ------------------------------ Abstract ---------------------------------- %%
\begin{abstract}
The goal of this paper is to develop a general covariant Hamiltonian approach to generalized harmonic formulation of general relativity. As en route investigations, an important class of coordinate transformation in the context of 3 + 1 decomposition, foliation preserving coordinate transformation, is defined; transformation rules of various 3 + 1 decomposition variables under this change of coordinates are investigated; the notion of covariant time derivative under foliation preserving coordinate transformation is defined; gauge conditions of various numerical relativity formulations are rewritten in general covariant form. The Hamiltonian formulation of generalized harmonic system is defined in the latter part of this paper. With the knowledge of covariant time derivative, the Hamiltonian formulation is extended to achieve general covariance. The Hamiltonian formulation is further proved to be symmetric hyperbolic. 
\end{abstract}


%% ---------------------------- Copyright page ------------------------------ %%
%% Comment the next line if you don't want the copyright page included.
\makecopyrightpage

%% -------------------------------- Title page ------------------------------ %%
\maketitlepage

%% -------------------------------- Dedication ------------------------------ %%
\begin{dedication}
 \centering To my parents and beloved wife \break who supported my pursuit of this unrealistic endeavor. 
\end{dedication}

%% -------------------------------- Biography ------------------------------- %%
\begin{biography}
Meng Cao (born 1986) joined the doctoral program of Department of Physics, North Carolina State University in autumn 2009. He began his study on general relativity in spring 2010 under Dr. John David Brown. During his five years in the program, he focused his research on various topics related to numerical relativity, including implementing nodal discontinuous Galerkin method to solve Regge-Wheeler/Zerilli equations and developing Hamiltonian approach to generalized harmonic formulation of general relativity. 
\end{biography}

%% ----------------------------- Acknowledgements --------------------------- %%
\begin{acknowledgements}
I express my sincere appreciation to  Dr. John David Brown for his consistent help during my five years of graduate study, both as an advisor and as a friend. 

I am also grateful to my department and graduate school for offering me five years of teaching assistantship so that I am able to bring food to my table everyday; to North Carolina State University for providing such great facilities, such as D. H. Hill/James B. Hunt Jr. Libraries and Carmichael gymnasium, for me to sustain a healthy mind and body. 
\end{acknowledgements}


\thesistableofcontents

%\thesislistoftables

%\thesislistoffigures
