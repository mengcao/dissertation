\documentclass[11pt,          % font size: 11pt or 12pt
               phd,           % degree:    ms or phd
               onehalfspacing % spacing: onehalfspacing or doublespacing
               ]{ncsuthesis}

%%----------------------------------------------------------------------------%%
%%------------------------------ Import Packages -----------------------------%%
%%----------------------------------------------------------------------------%%

\usepackage{booktabs}  % professionally typeset tables
\usepackage{amsmath}
\usepackage{textcomp}  % better copyright sign, among other things
\usepackage{xcolor}
\usepackage{lipsum}    % filler text
\usepackage{subfig}    % composite figures
\usepackage{mathrsfs}


%%----------------------------------------------------------------------------%%
%%---------------------------- Formatting Options ----------------------------%%
%%----------------------------------------------------------------------------%%
%%

%% -------------------------------------------------------------------------- %%
%% Disposition format -- any titles, headings, section titles
%%  These formatting commands affect all headings, titles, headings,
%%  so sizing commands should not be used here.
%%  Formatting options to consider are
%%     +  \sffamily - sans serif fonts.  Dispositions are often typeset in
%%                    sans serif, so this is a good option. 
%%     +  \rmfamily - serif fonts
%%     +  \bfseries - bold face
%\dispositionformat{\sffamily\bfseries}   % bold and sans serif
\dispositionformat{\bfseries}            % bold and serif

%% -------------------------------------------------------------------------- %%
%% Formatting for centered headings - Abstract, Dedication, etc. headings
%%  This is where one might put a sizing command.
%%  \MakeUppercase can be used to typeset all headings in uppercase.
\headingformat{\large\MakeUppercase}   % All letters uppercase
%\headingformat{\large}                % Not all uppercase
%\headingformat{\Large\scshape}        % Small Caps, used with serif fonts.

%% Typographers recommend using a normal inter-word space after
%% sentences. TeX's default is to add an wider space, but \frenchspacing
%% gives a normal spacing. Comment out the following line if you prefer
%% wider spaces between sentences.
\frenchspacing


%% -------------------------------------------------------------------------- %%
%%  Optional packages
%%    A number of compatible packages to improve the look and feel of
%%    your document are available in the file optional.tex 
%%    (For example, hyperlinks, fancy chapter headings, and fonts)
%% To use these options, uncomment the next line and see optional.tex
%%%  Optional Packages to consider.   These packages are compatible with
%%    ncsuthesis.  

%% -------------------------------------------------------------------------- %%
%% Fancy chapter headings
%%  available options: Sonny, Lenny, Glenn, Conny, Rejne, Bjarne
\usepackage[Sonny]{fncychap}

%%----------------------------------------------------------------------------%%
%% Hyperref package creates PDF metadata and hyperlinks in Table of Contents
%%  and citations.  Based on feedback from the NCSU thesis editor, 
%%  the links are not visually distinct from normal text (i.e. no change
%%  in color or extra boxes).
\usepackage[
  pdfauthor={John Mark Smith},
  pdftitle={The Title},
  pdfcreator={pdftex},
  pdfsubject={NC State ETD Thesis},
  pdfkeywords={keyword1, keyword2},
  colorlinks=true,
  linkcolor=black,
  citecolor=black,
  filecolor=black,
  urlcolor=black,
]{hyperref}


%% -------------------------------------------------------------------------- %%
%% Microtype - If you use pdfTeX to compile your thesis, you can use
%%              the microtype package to access advanced typographic
%%              features.  By default, using the microtype package enables
%%              character protrusion (placing glyphs a hair past the right 
%%              margin to make a visually straighter edge)
%%              and font expansion (adjusting font width slightly to get 
%%              more favorable justification).
%%              Using microtype should decrease the number of lines
%%              ending in hyphens.
\usepackage{microtype}


%%----------------------------------------------------------------------------%%
%% Fonts 

%% ETD guidelines don't specify the font.  You can enable the fonts
%%  by uncommenting the appropriate lines.  Using the default Computer 
%%  Modern fonts is *not* required.  A few common choices are below.
%%  See http://www.tug.dk/FontCatalogue/ for more options.

%% Serif Fonts -------------------------------------------------
%%  The four serif fonts listed here (Utopia, Palatino, Kerkis,
%%  and Times) all have math support.


%% Utopia
\usepackage[T1]{fontenc}
\usepackage[adobe-utopia]{mathdesign}

%% Palatino
%\usepackage[T1]{fontenc}
%\usepackage[sc]{mathpazo}
%\linespread{1.05}

%% Kerkis
%\usepackage[T1]{fontenc}
%\usepackage{kmath,kerkis}

%% Times
%\usepackage[T1]{fontenc}
%\usepackage{mathptmx}


%% Sans serif fonts -------------------------

%\usepackage[scaled]{helvet}  % Helvetica
%\usepackage[scaled]{berasans} % Bera Sans



%%----------------------------------------------------------------------------%%
%%---------------------------- Content Options -------------------------------%%
%%----------------------------------------------------------------------------%%
%% Size of committee: 3, 4, 5, or 6 -- this number includes the chair
\committeesize{4}

%% Members of committee
%%  Each of the following member commands takes an optional argument
%%   to specify their role on the committee.
%%  For co-chairs, use the commands:
%%      \cochairI{Doug Dodd}
%%      \cochairII{Chris Cox}
%%
\chair{John David Brown}
\memberI{Dean Lee}
\memberII{Lubos Mitas}
\memberIII{Ilse Ipsen}   % unnecessary if committeesize=3


%% Student writing thesis, \student{First Middle}{Last}
\student{Meng}{Cao} % a full middle name
%\student{John M.}{Smith} % a middle initial

%% Degree program
\program{Physics}

%% Thesis Title
%%  Keep in mind, according to ETD guidelines:
%%    +  Capitalize first letter of important words.
%%    +  Use inverted pyramid shape if title spans more than one line.
%%
%%  Note: To break the title onto multiple lines, use \break instead of \\.
\thesistitle{Generally Covariant Hamiltonian Approach \break to \break Generalized Harmonic Formulation of General Relativity}

%% Degree year.  Necessary if your degree year doesn't equal the current year.
\degreeyear{2014}


%%----------------------------------------------------------------------------%%
%%---------------------------- Personal Macros -------------------------------%%
%%----------------------------------------------------------------------------%%

%% A central location to add your favorite macros.

%% A few examples to get you started.
\newcommand{\uv}[1]{\ensuremath{\mathbf{\hat{#1}}}}
\newcommand{\bo}{\ensuremath{\boldsymbol{\Omega}}}
\newcommand{\eref}[1]{Eq.~\ref{#1}}
\newcommand{\fref}[1]{Figure~\ref{#1}}
\newcommand{\tref}[1]{Table~\ref{#1}}

%%---------------------------------------------------------------------------%%
\begin{document}

%%---------------------------------------------------------------------------%%
\frontmatter

%% ------------------------------ Abstract ---------------------------------- %%
\begin{abstract}
The goal of this paper is to develop a generally covariant Hamiltonian approach to generalized harmonic formulation of general relativity. As en route investigations, an important class of coordinate transformation in the context of 3 + 1 decomposition, foliation preserving transformation, is defined; transformation rules of various 3 + 1 decomposition variables under this change of coordinates are investigated; the notion of covariant time derivative under foliation preserving transformation is defined; gauge conditions of various numerical relativity formulations are rewritten in generally covariant form. The Hamiltonian formulation of generalized harmonic system is defined in the latter part of this paper. With the knowledge of covariant time derivative, the Hamiltonian formulation is extended to achieve general covariance. The Hamiltonian formulation is further proved to be symmetric hyperbolic. 
\end{abstract}


%% ---------------------------- Copyright page ------------------------------ %%
%% Comment the next line if you don't want the copyright page included.
\makecopyrightpage

%% -------------------------------- Title page ------------------------------ %%
\maketitlepage

%% -------------------------------- Dedication ------------------------------ %%
\begin{dedication}
 \centering To my parents and beloved wife \break who supported my pursuit of this unrealistic endeavor. 
\end{dedication}

%% -------------------------------- Biography ------------------------------- %%
\begin{biography}
Meng Cao (born 1986) joined the doctoral program of Department of Physics, North Carolina State University in autumn 2009. He began his study on general relativity in spring 2010 under Dr. John David Brown. During his five years in the program, he focused his research on various topics related to numerical relativity, including implementing nodal discontinuous Galerkin method to solve Regge-Wheeler/Zerilli equations and developing Hamiltonian approach to generalized harmonic formulation of general relativity. 
\end{biography}

%% ----------------------------- Acknowledgements --------------------------- %%
\begin{acknowledgements}
I express my sincere appreciation to  Dr. John David Brown for his consistent help during my five years of graduate study, both as an advisor and as a friend. 

I am also grateful to my department and graduate school for offering me five years of teaching assistantship so that I am able to bring food to my table everyday; to North Carolina State University for providing such great facilities, such as D. H. Hill/James B. Hunt Jr. Libraries and Carmichael gymnasium, for me to sustain a healthy mind and body. 
\end{acknowledgements}


\thesistableofcontents

%\thesislistoftables

\thesislistoffigures


%%---------------------------------------------------------------------------%%
\mainmatter

\chapter{Introduction}\label{introduction}
The purpose of this dissertation is to develop a Hamiltonian system for generalized harmonic formulation of general relativity. Our motivation of this specific topic is discussed in Sec.~\ref{background} after briefly reviewing Einstein's general theory of relativity and the significant role of Hamiltonian formulation in history of physics. Notation and convention followed throughout this disseratation are presented in Sec.~\ref{notation}. In the end of this chapter, abstracts of each following chapter is listed in Sec.~\ref{abstracts}.
%%%%%%%%%%%%%%%%%%%%%%%%%%%%%%%%%%%%
\section{Background Information and Motivation}\label{background}
\subsection{General Relativity}\label{gr}
During the years between 1911 and 1916, Einstein developed an elegant geometric approach to generalize Newtonian mechanics and his own special theory of relativity. The new theory formulated an innovative interpretation toward gravitation and was named general relativity. Other than most of natural forces, whose existence is represented by the field defined on spacetime, Einstein proposed in his generalized theory of relativity that gravitation is inherent in spacetime itself and it is the direct result of the curvature of spacetime\cite[pp. 1]{carroll2003spacetime}.

When his first paper on this field of study, {\it The Foundation of the Generalised Theory of Relativity}, was published in 1916, it was regarded as a heterodox theory packed with controversial and bold predictions that challenged people's knowledge. Its seemingly incompetence in applications was another obstacle for its adoption by the majority. In 1919, Sir Arthur Stanley Eddington, an astrophysicist from British Royal Society, observed and measured the bending angle of light by the sun to a very small vicinity of Einstein's estimate, which served as the first experimental evidence supported general relativity. Two of his other predictions, gravitational red shift and the precession of the perihelion of Mercury around the sun, were also confirmed by experiments consecutively. From then on, various consequences of general relativity, from the big bang, expanding universe to black holes, were accepted by the mainstream physics society and hence inspired huge amount of young people's interests in physics. After being studied and elaborated for several decades, now it is the standard description of gravitation in the framework of modern physics. 

Lying in the core of general relativity is a set of 10 partial differential equations, namely Einstein field equations or Einstein's equations, which governs properties of our spacetime geometry. Although there are several famous analytical solution of Einstein's equations, such as the Schwarzschild solution, it is unarguably a challenge to solve those equations manually for a wide variety of physical phenomena for which general relativity must account. Due to the prosperity of computer science, a new branch of general relativity, numerical relativity, emerged above the horizon. It employs numerical methods and algorithms, leverages power of high performance computers to solve Einstein's equations and hence to study black holes, gravitational waves, neutron stars and other phenomena governed by general relativity.

Numerical simulation also plays a significant role in the stage of experimental general relativity. Nearly 100 years ago, Sir Eddington used optical telescope and photographic plates to observe and measure light bent by the sun, thanks to the blooming technology, now we have multiple cutting-edge ground-based facilities stretching in miles listening to gravitational waves, such as LIGO, Virgo and GEO. Moreover, LISA/eLISA, the first space-based gravitational wave telescope, is under construction and is expected to reach a new level of accuracy for detecting gravitational waves. Waveform signals from these sophisticated gravitational wave detectors can be compared to numerical results to gain a better understanding of the gravitational radiation sources. On the other hand, numerical simulation can also produce artificial templates of gravitational waveforms. With the help of these templates, noise signals can be filtered out from experimental datas picked up by the detectors and hence to amplify the probability of detection. A detailed discussion of numerical relativity is presented in Chap.~\ref{nr}. 
%%%%%%%%%%%%%%%%%%%%%%%%%%%%%%%%%%%%%%%%%%%%%
\subsection{Hamiltonian Formulation}\label{ham}
Our study toward solving Einstein's equations takes the Hamiltonian viewpoint, rather than Lagrangian perspective. Lagrangian and Hamiltonian mechanics are two mathematically equivalent branches of classical mechanics. In Lagrangian formulation a system with $n$ degrees of freedom is described by $n$ second order partial differential equations with $2n$ initial conditions. The $2n$ initial conditions are usually specified by values of $n$ generalized coordinates $q_{i}$ and $n$ of $q_{i}$'s time derivative, ${\dot q}_{i}$ at time $t_{0}$. From Hamiltonian perspective, the same system with $n$ degrees of freedom is interpreted in $2n$ first order partial differential equations of $2n$ independent variables, while their values at time $t_{0}$ serve as $2n$ initial conditions. Those $2n$ independent variables are regarded as canonical variables and are usually constructed by $n$ generalized coordinates $q_{i}$ and their conjugate momentums $p_{i}$.

Comparing to Lagrangian procedure, Hamiltonian method doesn't particularly provide a superior solution to mechanics problems. Mathematically, the transition from Lagrangian to Hamiltonian formulation is merely an operation of changing the set of variables $(q_{i}, {\dot q}_{i}, t)$ to $(q_{i}, p_{i}, t)$, namely Legendre transformation. Furthermore, a system's equations of motion from Hamiltonian perspective are practically the same partial differential equations provided by Lagrangian formulation. The power and advantages of Hamiltonian formulation lies in providing a framework for theoretical extensions in many area of physics. With a deeper insight into mechanics, Hamiltonian formulation treats coordinates and momenta equally as independent variables, which entitles physicists to choose any arbitrary yet appropriate physical quantities as coordinates and momenta. A broader selection of dynamic variables leads to a more abstract way of presenting applications to mechanical problems, which plays an essential role in constructing more modern theories. It is Hamiltonian formulation who served as an embarking point from classical mechanics to both statistical mechanics and quantum theory in the past century and it is why it still attracts attentions from the physics society\cite{goldstein}. 
%%%%%%%%%%%%%%%%%%%%%%%%%%%%%%%%%%%%%%
\subsection{Motivation}
The dynamic development of numerical relativity and advantages of Hamiltonian formalism discussed in Sec.~\ref{gr} and Sec.~\ref{ham} motivate us to develop a numerical formulation to solve Einstein's equations in Hamiltonian perspective. While there already exist numerical algorithms yield satisfactory solutions to Einstein field equations, our objective is to provide an alternative approach to the investigation of numerical relativity. Meanwhile, we also hope our Hamiltonian interpretation of Einstein's equations could inspire innovative insights and understandings of general relativity. 
%%%%%%%%%%%%%%%%%%%%%%%%%%%%%%%%%%%%%
\section{Notation and Convention}\label{notation}
Inevitably, we need to work with tensors while studying general relativity. For tensor representation, we use Greek letter indices $\mu$, $\nu$, ... to denote spacetime indices from 0 to 3 and Latin letters $a$, $b$, ... for spatial indices from 1 to 3. In places where ambiguity might arise, we use a prefix $^{(4)}$ to denote a spacetime tensor and hence to distinguish it from its spatial counterpart, as in $^{(4)}g_{\mu\nu}$ or $^{(4)}R_{\mu\nu}$. In addition, we employ geometrized  units system throughout this dissertation, in which $G = c = 1$.

Ordinary partial derivative is denoted as a shorthand operator $
\partial_{\mu} \equiv \frac{\partial}{\partial x^{\mu}}$. Sometimes for the purpose of a clean presentation, we also use a single dot notation to denote ordinary time derivative for arbitrary variable $v$, i.e., ${\dot v} \equiv \partial_{t}v$. $\nabla_{\mu}$ stands for spacetime covariant derivative while $D_{a}$ is spatial covariant derivative. And $D_{t}$ represents the covariant time derivative which is discussed later in this dissertation. Lie derivative along a generic vector $v$ is denoted as $\mathcal{L}_{v}$. 

In this dissertation, we follow the convention used by Minser, Thorne and Wheeler\cite{Misner:1974qy}, namely the ``Landau-Lifshitz Spacelike Convertion'' (LLSC). Specifically, signature of spacetime metric $g_{\mu\nu}$ is ( --, +, +, + ), Riemann tensor is defined as 
\begin{equation}\label{riemanntensor}
	R^{\alpha}_{~\beta \mu \nu} \equiv 
	\partial_{\mu}\Gamma^{\alpha}_{~\beta \nu} - 
	\partial_{\nu}\Gamma^{\alpha}_{~\beta \mu} + \Gamma^{\alpha}_{~\sigma \mu}\Gamma^{\sigma}_{~\beta \nu} - \Gamma^{\alpha}_{~\sigma \nu}\Gamma^{\sigma}_{~\beta\mu} \ .
\end{equation}
In Eq.~(\ref{riemanntensor}), $\Gamma^{\alpha}_{~\mu\nu}$ stands for the Christoffel symbols, or sometimes regarded as the connection. The Christoffel symbols help to describe how parallel transport is defined in curved manifolds. Its mathematical definition is as follows, 
\begin{equation}
	\Gamma^{\alpha}_{~\mu\nu} \equiv \frac{1}{2}g^{\alpha\beta}\left(\partial_{\mu}g_{\beta\nu} + \partial_{\nu}g_{\beta\mu} - \partial_{\beta}g_{\mu\nu}\right) \ .
\end{equation}
Contracting two indices of the Riemann tensor gives us the Ricci tensor as below
\begin{equation}
	R_{\mu\nu} \equiv R^{\sigma}_{~\mu\sigma\nu} \ ,  
\end{equation}
while the Ricci scalar $R$ is the trace of Ricci tensor, 
\begin{equation}
	R \equiv R_{\mu\nu}g^{\mu\nu}\ .
\end{equation}
With all these definitions listed above, the Einstein field equations are stated as 
\begin{equation}
	G_{\mu\nu} \equiv R_{\mu\nu} - \frac{1}{2} R g_{\mu\nu} = 8\pi T_{\mu\nu} 
\end{equation}
and $T_{\mu\nu}$ is the stress-energy tensor. 

\section{Chapter Abstracts}\label{abstracts}
The structure of the rest of this dissertation is as follows. Chap.~\ref{nr} reviews the development of numerical relativity; presents the foundation of solving Einstein's equations numerically: the $3 + 1$ decomposition; discusses several mainstream numerical formulations and reveals their limitations. Chap.~\ref{generalcovariance} studies an important class of coordinate transformation in the context of $3 + 1$ decomposition. General covariance of different numerical formalisms under this type of coordinate transformation is achieved by investigating transformation rules of various field variables and introducing the notion of covariant time derivatives. A Hamiltonian system of generalized harmonic formulation is developed in Chap.~\ref{hamiltonian}. With the knowledge about covariant time derivatives obtained in Chap.~\ref{generalcovariance}, the Hamiltonian system is extended to reach its general covariance. Such a system is further proved to be well-posed, i.e., symmetric hyperbolic, at the end of that chapter. Chap.~\ref{summary} reviews this dissertation and gives suggestions about possible future development of this project. 

\chapter{Numerical Relativity}\label{nr}
As mentioned before, the rise of numerical relativity answered the need for a wide class of general solutions to Einstein's equations for various physical phenomena, such as perturbed black holes, coalescence of black holes and neutron stars, which general relativity must explain. As Wald pointed out,``If a corresponding large class of solutions of Einstein's equation failed to exist, we would be forced to reject general relativity as a correct theory of nature''\cite{Wald:GRbook}. Inevitably, in order to employ computers to tackle dynamical evolution problems, we have to separate the time and space dimensions Einstein united in his theory, i.e. convert Einstein's equations into an initial value problem. The 3 + 1 decomposition scheme to split time and space was initially proposed by Arnowitt, Deser and Misner\cite{ADM:Witten} and a detailed discussion of it is presented in Sec.~\ref{3+1}. Mainstream numerical formulations of general relativity are presented in Sec.~\ref{numericalformulation} and their limitations are discussed. 
\section{3 + 1 Decomposition}\label{3+1}
For initial value problems of ordinary particle mechanics, the evolution of a system is dependent on the initial conditions and equations of motion for the system. It is easy to feed computers with initial value data and integrate the partial differential equations of motion along the time axis numerically to obtain the dynamic evolution of the system. However, since Einstein united time and space in his theory of gravity, initial value data is not native in Einstein's equations. Therefore, an important prerequisite for developing any numerical formulation for the Einstein field equations is to decompose the four dimensional spacetime into a one-parameter family of nonintersecting spacelike hypersurfaces, namely a foliation, ``one slice for each instant of time''. The operation of constructing a rigid structure of such slices is referred as ``3 + 1 decomposition'' and it is demonstrated in Fig.~\ref{3+1diagram} (a thorough discussion of 3 + 1 decomposition can be found in Ref.~\cite{Gourgoulhon:2007ue}). 

\begin{figure}[hbtp]
\centering
\includegraphics{decomposition-1}
\caption{A spacetime diagram illustrating the 3 + 1 decomposition. $t^{\mu}$ is the flow of time vector; $n^{\mu}$ is the normal vector on the lower slice; $g_{ab}$ is the spatial metric for the lower slice; $\alpha$ is regarded as the lapse function while $\beta^{a}$ is known as the shift vector. }
\end{figure}\label{3+1diagram}

In order to  construct a rigid structure, several physical quantities need to be specified. First and foremost is the three dimensional spatial metric $g_{ab}$ on each slice, which defines the proper distance between two points residing on the hypersurface. Second of all, the proper distance between each slice is specified by the lapse function $\alpha$. Although it is not necessary to keep the coordinate system on successive slices related to each other, due to practical considerations, a shift vector $\beta^{a}$ is defined to weld the coordinate systems between consecutive hypersurfaces. The spacetime metric $^{(4)}g_{\mu\nu}$ is split in terms of the spatial metric $g_{ab}$, lapse function $\alpha$ and shift vector $\beta^{a}$ as follows
\begin{equation}
	{}^{\left(4\right)}g_{\mu\nu} = \left(-\alpha^2 + \beta^{a}\beta_{a}\right)\delta_\mu^0\delta_\nu^0 
	+ 2\beta_{a}\delta_{(\mu}^a\delta_{\nu)}^0 + g_{ab} \delta^a_\mu \delta^b_\nu \ ,
\end{equation}
where $\delta^{\mu}_{\nu}$ is the usual delta tensor. 

The covariant normal to each foliation is denoted as
\begin{equation}\label{normal covector}
n_{\mu} \equiv -\alpha\delta^{0}_{\mu} \ ,
\end{equation}
and its dual is
\begin{equation}\label{normal vector}
n^{\mu} \equiv \left(\delta^{\mu}_{0} - \beta^{c}\delta^{\mu}_{c}\right)/\alpha
\end{equation}
so that we have $n_{\mu}n^{\mu} = -1$. 

There is also the projection operator
\begin{equation}\label{projection 1}
X^{\mu}_{a} \equiv \delta^{\mu}_{a}
\end{equation}
that projects a spacetime covector into a spacelike covector. Its covariant form is
\begin{equation}\label{projection 2}
X^{a}_{\mu} \equiv \delta^{a}_{\mu} + \beta^{a}\delta^{0}_{\mu}
\end{equation}
so that we have $X^{\mu}_{a}X^{b}_{\mu} = \delta^{b}_{a}$ and $X^{\mu}_{a}n_{\mu} = 0$. 

With these definitions, the spacetime metric can be written in terms of the spatial metric, the normal and the projection operator as
\begin{equation}\label{spacetime metric 3 + 1}
^{\left(4\right)}g_{\mu\nu} = g_{ab}X^{a}_{\mu}X^{b}_{\nu} - n_{\mu}n_{\nu}
\end{equation}
Spacetime indices $\mu$, $\nu$, ... are always raised and lowered with the spacetime metric $^{\left(4\right)}g_{\mu\nu}$ and its inverse $^{\left(4\right)}g^{\mu\nu}$, while spatial indices $a$, $b$, ... are always raised and lowered with the spatial metric $g_{ab}$ and its inverse $g^{ab}$. 

The notion of extrinsic curvature in the 3 + 1 context represents the time derivative of the spatial metric for the foliation. It is defined as
\begin{equation}\label{extrinsic}
K_{ab} = - \frac{1}{2\alpha}(\partial_{t} - \mathcal{L}_{\beta})g_{ab} \ ,
\end{equation}
and it is easy to show that $K_{ab}$ is a symmetric tensor, i.e., $K_{ab} = K_{ba}$. Physically, the extrinsic curvature, as an abstract coordinate-independent geometric object, defines the curvature of a three dimensional hypersurface relative to the four dimensional spacetime manifold it is embedded in. Imagine parallel transporting the normal covector $n_{\mu}$ of slice $\Sigma$ from point $p$ to point $q$. This parallel transported covector will fail to coincide with the normal covector $n_{\mu}$ at $q$. This failure reflects the bending of $\Sigma$ in spacetime and the extrinsic curvature directly measures the deviation between the parallel transported covector and $n_{\mu}$ at $q$. 

With the help of the 3 + 1 decomposition, one is able to specify a set of dynamic variables ( $g_{ab}$, $K_{ab}$, etc. ) on one slice and utilize it as the initial condition. Different variations of the Einstein field equations have been developed to integrate the initial data along time and obtain a uniquely determined dynamical evolution of the spacetime geometry. Several important types of  numerical formulations are discussed in Sec.~\ref{numericalformulation}.  
%%%%%%%%%%%%%%%%%%%%%%%%%%%%%%%%%%%%%%%%%%%%%%%%%

%%%%%%%%%%%%%%%%%%%%%%%%%%%%%%%%%%%%%%%%%%%%%%%
\section{Numerical Formulations of General Relativity}\label{numericalformulation}
This section briefly reviews the development of numerical formulations for solving Einstein's equations while detailed review of this topic can be found in Ref.~\cite{lrr-2012-9}. Two widely adopted modern numerical formalisms are discussed in detail. Our motivation for this work is reviewed at the end of this section. 

\subsection{ADM and BSSN formalisms}
The foundation for solving Einstein field equations numerically was laid down by Arnowitt, Deser and Misner via their famous ADM formalism\cite{ADM:Witten} in 1959. In the original ADM paper, a Hamiltonian approach was taken toward rewriting Einstein's equations. The Hilbert-Palatini gravitational action was written in terms the spatial metric $g_{ab}$, its conjugate momentum $\pi^{ab}$, lapse function $\alpha$ and shift vector $\beta^{a}$. The dynamic variables are chosen to be $g_{ab}$ and $\pi^{ab}$. Variation of the action with respect to the spatial metric and its momentum yields evolution equations for $g_{ab}$ and $\pi^{ab}$, while varying the action with respect to the lapse function and shift vector provides constraint equations for the initial conditions of the dynamic variables. These equations, obtained by varying the action, determine how the system evolves with time. 

In the numerical system now commonly referred to as the ADM formalism\cite{Smarr:York,Smarr:1977uf}, the extrinsic curvature $K_{ab}$ replaces $\pi^{ab}$ as one of the dynamic variables and they are closely related as $\pi^{ab} = -\sqrt{g}(K^{ab} - g^{ab}K)$. The evolution equations of $g_{ab}$ and $K_{ab}$ drive the system dynamically. Two sets of constraint equations known as the Hamiltonian and momentum constraints govern the choice of initial data for the spatial metric and extrinsic curvature. These equations together are equivalent to the Einstein field equations. 

Unfortunately, the ADM formalism was later proved to be impractical for the construction of numerical simulations because the system of equations is only weakly hyperbolic. As mentioned in Chap.~\ref{wellposedness}, weak hyperbolicity guarantees the ill-posedness of the system while strong hyperbolicity is a necessary condition for the system to be well-posed in the absence of spatial boundaries. Furthermore, symmetric hyperbolicity is a necessary condition for the system to be well-posed in the presence of spatial boundaries. The notion of well-posedness is discussed in detail in Chap.~\ref{wellposedness}. Essentially well-posedness defines the system's analytical stability under physical perturbations of the initial data. This means that for a ill-posed system, such as the ADM equations, small changes in the initial data might lead to non-proportional changes of the solution and this is the direct reason of ADM formulation's incompetence for numerical application. 

During the years between 1987 to 1999 Baumgarte, Shapiro, Shibata and Nakamura\cite{Shibata:1995we, Baumgarte:1998te} developed a strongly hyperbolic modification of the ADM formalism. This modified system is known as BSSN formalism and is one of the most commonly used numerical formulation for relativity applications. 

Instead of evolving the spatial metric and extrinsic curvature, the BSSN formalism chooses a conformal factor $\phi$, the trace of extrinsic curvature $K \equiv K_{ab}g^{ab}$ and a set of conformal variables, namely the conformal spatial metric ${\tilde g}_{ab} \equiv e^{-4\phi}g_{ab}$, conformal trace-free extrinsic curvature ${\tilde A}_{ab} \equiv e^{-4\phi}(K_{ab} - \frac{1}{3}g_{ab}K)$ and conformal connection functions ${\tilde \Gamma}^{a} \equiv {\tilde g}^{bc}{\tilde \Gamma}^{a}_{~bc}$, as the fundamental variables to evolve along with time. Supplemented with the moving puncture gauge conditions which govern evolution of the lapse function and shift vector, BSSN formalism is widely used in the numerical relativity community to analyze a variety of gravitational problems. The BSSN equations are strongly hyperbolic. 

\subsection{Generalized Harmonic Formulation}
Another well recognized numerical formulation of Einstein's equations is the generalized harmonic(GH) equations\cite{Friedrich:1985, Garfinkle:2001ni, Pretorius:2006tp}. As many numerical algorithms are recognized as stable and efficient when dealing with wave equations, the GH formulation writes Einstein field equations in a form which is similar to curved spacetime wave equations by utilizing harmonic coordinates. The generalized harmonic coordinates used by the GH formulation satisfy the wave equations
\begin{equation}
	\nabla^{\alpha}\nabla_{\alpha}x^{\mu} = H^{\mu} \ ,
\end{equation}
where $H^{\mu}$ is an freely chosen source vector\cite{Friedrich:1985, Friedrich:1996, Garfinkle:2001ni}.

The GH version of Einstein field equations can be expressed in an abstract form as
\begin{equation}\label{ghequations}
	g^{\alpha\beta}\partial_{\alpha}\partial_{\beta}g_{\mu\nu} + \partial_{(\mu}H_{\nu)} = Q_{\mu\nu}(H, g, \partial g) \ , 
\end{equation}
where $Q_{\mu\nu}$ are functions that depend on the source vector, spacetime metric and its first-order derivatives. Equations for the source terms $H^{\mu}$ are evolved along with Eqs.~\ref{ghequations} to solve Einstein's equations numerically. These equations are referred to as gauge conditions. Gauge conditions\cite{Lindblom:2007xw, Lindblom:2009tu} have been developed to combine with Eqs.~(\ref{ghequations}) such that the complete system of equations is symmetric hyperbolic\cite{Lindblom:2005qh}. 

In recent years, the 3 + 1 form of the GH equations\cite{Brown:2011qg} has been developed
to reveal the relationship between the GH, ADM and BSSN formulations. Furthermore, the action principle for the GH formulation\cite{Brown:2010rya} is presented to provide an alternative perspective to understanding this formalism. The foundation laid down by these two pieces of work and the significant role that the Hamiltonian perspective has played in history of physics directly motivate us to develop the Hamiltonian system of the GH equations. 

During our research on the Hamiltonian approach to the GH formalism, we found that despite their supremacy in practical applications, both BSSN and GH formulations have some drawbacks regarding general covariance. This issue is addressed in Chap.~\ref{generalcovariance} before we move on to the Hamiltonian formulation. 


\chapter{Numerical Relativity}\label{nr}
As mentioned before, the rise of numerical relativity answered for the desire of a wide class of general solution to Einstein's equations for various physical phenomena, such as perturbed black holes, coalescence of black holes and neutron stars, which general relativity must explain. As Wald pointed out,``If a corresponding large class of solutions of Einstein's equation failed to exist, we would be forced to reject general relativity as a correct theory of nature''\cite{Wald}. Inevitably, in order to employ computers to tackle dynamical evolution problems, we have to separate time and space dimensions Einstein united in his theory, i.e. convert Einstein's equations into an initial value problem. The 3 + 1 decomposition scheme to split time and space was initially proposed by Arnowitt, Deser and Misner\cite{ADM:Witten} and a detailed discussion of it is presented in Sec.~\ref{3+1}. Another important aspect of numerical formulation, stability and accuracy, is addressed in Sec.~\ref{wellposedness} in the context of well-posedness. Main stream numerical formulations of general relativity are presented in Sec.~\ref{numericalformulation} and their limitations are discussed. 
%\section{Initial Value Problem}\label{initial}
\section{3 + 1 Decomposition}\label{3+1}
For initial value problems of ordinary particle mechanics, the evolution of a system is dependent on the initial conditions and equations of motion for the system. It is easy to feed computers with initial value data and integrate the partial differential equations of motion along time axis numerically to obtain dynamic evolution of the system. During the past several decades, various types of numerical algorithms are developed to solve initial value problems and they are proved to be successful. However, since Einstein united time and space dimensions in his theory of gravity, initial value data is not native in Einstein's equations. Therefore, an important prerequisite for developing any numerical formulation is to decompose the four dimensional spacetime into a one-parameter family of nonintersecting spacelike hypersurfaces, namely foliations, ``one for each instant of time''. The operation of constructing a rigid structure of such series of foliations is referred as ``3 + 1 decomposition''. 

In order to  construct a rigid structure of the decomposition, several physical quantities need to be specified. First and foremost is the three dimensional spatial metric $g_{ab}$ on each foliation, which defines the proper distance between two points reside on the foliation. Second of all, the proper distance between each foliation is specified by the lapse function $\alpha$. Although it is not necessary to keep coordinate system on each foliation related to each other, due to practical consideration, a shift vector $\beta^{a}$ is defined to weld coordinate systems between each two consecutive hypersurfaces. The spacetime metric $^{(4)}g_{\mu\nu}$ is split in terms of spatial metric $g_{ab}$, lapse function $\alpha$, shift vector $\beta^{a}$ as following
\begin{equation*}
	{}^{\left(4\right)}g_{\mu\nu} = \left(-\alpha^2 + \beta^{a}\beta_{a}\right)\delta_\mu^0\delta_\nu^0 
	+ 2\beta_{a}\delta_{(\mu}^a\delta_{\nu)}^0 + g_{ab} \delta^a_\mu \delta^b_\nu \ ,
\end{equation*}
where $\delta^{\mu}_{\nu}$ is regular delta tensor. 

The covariant normal to each foliation is denoted as
\begin{equation}\label{normal covector}
n_{\mu} = -\alpha\delta^{0}_{\mu}
\end{equation}
 And its dual is
\begin{equation}\label{normal vector}
n^{\mu} = \left(\delta^{\mu}_{0} - \beta^{c}\delta^{\mu}_{c}\right)/\alpha
\end{equation}
so that we have $n_{\mu}n^{\mu} = -1$. 

There is also the projection operator
\begin{equation}\label{projection 1}
X^{\mu}_{a} = \delta^{\mu}_{a}
\end{equation}
that projects a spacetime covector into a spacelike covector. Its covariant form is
\begin{equation}\label{projection 2}
X^{a}_{\mu} = \delta^{a}_{\mu} + \beta^{a}\delta^{0}_{\mu}
\end{equation}
so that we have $X^{\mu}_{a}X^{b}_{\mu} = \delta^{b}_{a}$ and $X^{\mu}_{a}n_{\mu} = 0$. 

With these definitions, the spacetime metric can be written in terms of the spatial metric, the normal and the projection operator as
\begin{equation}\label{spacetime metric 3 + 1}
^{\left(4\right)}g_{\mu\nu} = g_{ab}X^{a}_{\mu}X^{b}_{\nu} - n_{\mu}n_{\nu}
\end{equation}
Spacetime indices $\mu$, $\nu$, ... are always raised and lowered with the spacetime metric $^{\left(4\right)}g_{\mu\nu}$ and its inverse $^{\left(4\right)}g^{\mu\nu}$, while spatial indices $a$, $b$, ... are always raised and lowered with the spatial metric $g_{ab}$ and its inverse $g^{ab}$. 

The notion of extrinsic curvature in 3 + 1 decomposition context represents a well-defined time derivative of the spatial metric on the spacelike hypersurface. It is defined as
\begin{equation}\label{extrinsic}
K_{ab} = - \frac{1}{2\alpha}(\partial_{t} - \mathcal{L}_{\beta})g_{ab} \ ,
\end{equation}
and it is easy to show that $K_{ab}$ is a symmetric tensor, i.e., $K_{ab} = K_{ba}$. Physically, the extrinsic curvature, as an abstract coordinate-independent geometric object, defines curvature of a three dimensional foliation relative to the four dimensional spacetime manifold it is embedded in. Imagine parallel transporting the normal covector $n_{a}$ of foliation $\Sigma$ from point $p$ to point $q$. This parallel transported covector will fail to coincide with the normal covector $n_{a}$ on $q$. This failure reflects the bending of foliation $\Sigma$ in spacetime and extrinsic curvature directly measures the deviation between the parallel transported covector and $n_{a}$ at $q$. 

With the help of 3 + 1 decomposition, one is able to specify a set of dynamic variables ( $g_{ab}$, $K_{ab}$, etc. ) on one foliation and utilize it as the initial condition. Different variations of Einstein field equations were developed by various classes of numerical relativity formulations to integrate the initial condition along time and obtain a uniquely determined dynamic evolution of the spacetime geometry in interest. Several of the mainstream numerical formulations are discussed in Sec.~\ref{numericalformulation}, but before that we first need to talk about which properties a proper initial value numerical formalism should satisfy and the most important one is well-posedness.  
%%%%%%%%%%%%%%%%%%%%%%%%%%%%%%%%%%%%%%%%%%%%%%%%%
\section{Well-posedness}\label{wellposedness}
The notion of ``well-posed'' is originated from Jacques Hadamard, a French mathematician. He argues that mathematical models of physical phenomena should have the properties that a unique solution exists and the solution's behavior changes continuously with the initial conditions. Wald gave his own yet similar definition of well-posedness in his book General Relativity. Wald's definition states that in order for a numerical formulation to be well-posed, firstly small changes in initial data should produce only correspondingly small changes in the solution over any fixed compact region of spacetime; and secondly changes of initial data in a region should only be responsible for changes in the solution inside the causal future of this region. The first requirement is necessary since it is impossible to measure initial condition in infinite accuracy. The second property guarantees no signal could propagate faster than the speed of light and keep the numerical formulation under framework of relativity theory. 

The following section briefly reviews the development of numerical formulations for solving Einstein's equations. The generalized harmonic formalism for general relativity is discussed in detail and its limitations and restrictions are revealed. 
\section{Numerical Formulation of General Relativity}\label{numericalformulation}
\chapter{Hamiltonian Approach to Generalized Harmonic Formulation}\label{hamiltonian}
Einstein-Hilbert action encodes complete dynamic information of general relativity system. Applying variational principle to the action yields Einstein field equations. Both Lagrangian and Hamiltonian equations can be derived from the principle of least action. In this chapter, we write down the Einstein-Hilbert action of generalized harmonic formulation in Sec.~\ref{action}. Hamiltonian and equations of motion for GH system are spelled out in Sec.~\ref{hamiltonian}. An extension of the Hamiltonian for the purpose of general covariance is discussed in Sec.~\ref{extension}. In the end, a well-posedness analysis of the system is presented in Sec.~\ref{wellposedness}. 
%%%%%%%%%%%%%%%%%%%%%%%%%%%
%%%%%%%%%%%%%%%%%%%%%%%%%%%
\section{Action of Generalized Harmonic Formulation}\label{action}
As presented in Ref.~\cite{Brown:2010rya}, the Lagrangian for generalized harmonic gravity is the following function of the spacetime metric $g_{\mu\nu}$ and the gauge source vector $H_{\mu}$, 
\begin{equation}
\mathscr{L}\left(g_{\mu\nu}, H^{\mu}\right) = \sqrt{-^{(4)}g} \left(^{(4)}R - \frac{1}{2}C_{\mu}C^{\mu}\right) \ , 
\end{equation}
and the action is a functional of $g_{\mu\nu}$ and $H^{\mu}$
\begin{equation}\label{action}
S\left[g_{\mu\nu}, H^{\mu}\right] = \int \mathscr{L} d^{4}x \ .
\end{equation}
In the Lagrangian, $^{(4)}g$ is determinant of the spacetime metric, $^{(4)}R$ stands for spacetime Ricci scalar and $C_{\mu}$ is the generalized harmonic constrains defined as
\begin{equation}
C_{\mu} = H_{\mu} + \Delta \Gamma^{~~~\beta}_{\mu\beta}
\end{equation}
where $H_{\mu}$ is the gauge source vector. 

Employing 3 + 1 decomposition to split the spacetime metric $g_{\mu\nu}$ into spatial metric $g_{ab}$, lapse function $\alpha$ and shift vector $\beta^{a}$, the gauge source vector $H_{\mu}$ into its time-like component $H_{\perp}$ and spatial component $H_{a}$, the action turns into the following 3 + 1 form
\begin{equation}\label{3+1action}
S\left[g_{ab}, \alpha, \beta^{a}, H_{\perp}, H^{a}\right] = \int d^{4}x~~\alpha \sqrt{g} \left( R + K^{ab}K_{ab} - K^{2} - \frac{1}{2}C^{a}C_{a} + \frac{1}{2}C_{\perp}^{2}\right).
\end{equation}
In Eq.~(\ref{3+1action}), $R$ is the spatial Ricci scalar, $K$ is the trace of extrinsic curvature and $\mathcal{L}_{\beta}$ is the Lie derivative along the shift vector. 

The splitting of $C_{\mu}$ is more delicate. According to the 3 + 1 splitting discussed in Appendix A of Ref.~\cite{Brown:2011qg}, we split $C_{\mu}$ into $C_{\perp}$ and $C_{a}$, where
\begin{equation}
C_{\perp} = {\tilde H}_{\perp} + K + \frac{1}{\alpha^{2}}D_{t}\alpha
\end{equation}
and
\begin{equation}
C_{a} = {\tilde H}_{a} + \Delta \Gamma^{b}_{cd}g^{cd}g_{ab} - \frac{\partial_{a}\alpha}{\alpha} - \frac{g_{ab}}{\alpha^2}D_{t}\beta^{b} \ .
\end{equation}
Here, ${\tilde H}_{\perp}$ and ${\tilde H}_{a}$ are the modified gauge source vector components, where
\begin{subequations}
\begin{align}
{\tilde H}_{\perp} & \equiv H_{\perp} - \frac{\alpha}{{\bar \alpha}}{\bar K}_{ab}g^{ab} + \frac{{\bar K}_{ab}}{\alpha{\bar \alpha}}\Delta \beta^{a} \Delta \beta^{b} + \frac{2}{\alpha {\bar \alpha}}\Delta \beta^{a} \partial_{a}{\bar \alpha}, \\
{\tilde H}_{a} & \equiv H_{a} + \frac{{\bar \alpha}}{\alpha^2}g_{ab}{\bar g}^{bc}\partial_{c}{\bar \alpha} + \frac{g_{ab}\Delta \beta^{b}}{\alpha^2 {\bar \alpha}}(\alpha^2g^{cd}{\bar K}_{cd} - 2\Delta\beta^{c}\partial_{c}{\bar \alpha}) - \frac{g_{ab}{\bar K}_{cd}\Delta \beta^{c}}{\alpha^2{\bar \alpha}}(\Delta \beta^{b}\Delta \beta^{d} - 2{\bar \alpha}^2{\bar g}^{bd}) \ .
\end{align}
\end{subequations}
Since absorbing terms that depend on physical fields $g_{ab}$, $\alpha$, $\beta^{a}$, background fields ${\bar g}_{ab}$, ${\bar \alpha}$, ${\bar \beta}^{a}$ and derivatives of the background fields into $H_{\perp}$ and $H_{a}$ won't change the hyperbolicity of GH formulation\cite{Brown:2011qg}, later in the paper, we refer ${\tilde H}_{\perp}$ and ${\tilde H}_{a}$ as $H_{\perp}$ and $H_{a}$, respectively. Therefore, we rewrite the splitting of $C_{\perp}$ and $C_{a}$ as following
\begin{subequations}
\begin{align}
C_{\perp} & = H_{\perp} + K + \frac{1}{\alpha^{2}}D_{t}\alpha\\
C_{a} & = H_{a} + \Delta \Gamma^{b}_{cd}g^{cd}g_{ab} - \frac{\partial_{a}\alpha}{\alpha} - \frac{g_{ab}}{\alpha^2}D_{t}\beta^{b} \ .
\end{align}
\end{subequations}
With the action defined explicitly, we are now able to construct the Hamiltonian formalism of GH formulation in the following sections. 
%%%%%%%%%%%%%%%%%%%%%%%%
%%%%%%%%%%%%%%%%%%%%%%%%
\section{Hamiltonian and Equations of Motion}\label{hamiltonian}
According to Eq.~(\ref{3+1action}), we have that the Lagrangian in 3 + 1 form is
\begin{equation}
\mathscr{L}\left(g_{ab}, \alpha, \beta^{a}, H_{\perp}, H^{a}\right) = \alpha \sqrt{g} \left( R + K^{ab}K_{ab} - K^{2} - \frac{1}{2}C^{a}C_{a} + \frac{1}{2}C_{\perp}^{2}\right) \ .
\end{equation}
This allows us to define conjugate momenta of dynamical variables $g_{ab}$, $\alpha$, $\beta^{a}$, $H_{\perp}$ and $H^{a}$ as following
\begin{subequations}\label{momenta}
\begin{align}
P^{ab} & \equiv \frac{\partial \mathscr{L}}{\partial {\dot g}_{ab}} = \sqrt{g}\left(Kg^{ab} - K^{ab} - \frac{C_{\perp}}{2}g^{ab}\right)\\
\pi & \equiv \frac{\partial \mathscr{L}}{\partial {\dot \alpha}} = \frac{\sqrt{g}}{\alpha}C_{\perp}\\
\rho_{a} & \equiv \frac{\partial \mathscr{L}}{\partial {\dot \beta}^{a}} = \frac{\sqrt{g}}{\alpha}C_{a}\\
\Omega & \equiv \frac{\partial \mathscr{L}}{\partial {\dot H}_{\perp}} = 0\\
\Omega_{a} & \equiv \frac{\partial \mathscr{L}}{\partial {\dot H}^{a}} = 0 \ .
\end{align}
\end{subequations}

By definition of Hamiltonian density, we have
\begin{equation}
\mathscr{H} \equiv P^{ab}{\dot g}_{ab} + \pi{\dot \alpha} + \rho_{a}{\dot \beta}^{a} + \Omega {\dot H}_{\perp} + \Omega_{a}{\dot H}^{a} - \mathscr{L} \ . 
\end{equation}
To obtain the explicit expression of $\mathscr{H}$ in terms of $g_{ab}$, $\alpha$, $\beta^{a}$, $H_{\perp}$, $H^{a}$ and their conjugate momenta $P^{ab}$, $\pi$, $\rho_{a}$, $\Omega$, $\Omega_{a}$ we need to invert Eqs.~(\ref{momenta}) to eliminate ${\dot g}_{ab}$, ${\dot \alpha}$ and ${\dot \beta}^{a}$. The inversion can be written down as follows
\begin{subequations}
\begin{align}
D_{t}g_{ab} & = \frac{2\alpha P_{ab}}{\sqrt{g}} - \frac{\alpha P g_{ab}}{\sqrt{g}} - \frac{\alpha^{2}\pi g_{ab}}{2\sqrt{g}}\\
D_{t}\alpha & = \frac{\alpha^{3}\pi}{4\sqrt{g}} - \alpha^{2}H_{\perp} - \frac{\alpha^{2}P}{2\sqrt{g}}\\
D_{t}\beta^{a} & = \alpha^{2}H^{a} + \alpha^{2}\Delta \Gamma^{a}_{~bc}g^{bc} - \alpha g^{ab}\partial_{b}\alpha - \frac{\alpha^{3}}{\sqrt{g}}\rho^{a} \ .
\end{align}
\end{subequations}
It is worth noting that terms on the left hand side are covariant time derivative of $g_{ab}$, $\alpha$ and $\beta^{a}$, which are defined in Eq.~(\ref{spatialmetriccovarianttimederivative}, \ref{lapsecovarianttimederivative}, \ref{shiftcovarianttimederivative}), so that the equations are in a general covariant form. 

However, it is impossible to invert the last two equations in Eqs.~(\ref{momenta}), $\Omega = 0$ and $\Omega^{a} = 0$. They turn out to serve as primary constraints of the system and we need to write ${\dot H}_{\perp}$ and ${\dot H}_{a}$ as arbitrary multipliers of these constraints
\begin{subequations}
\begin{align}
{\dot H}_{\perp} & = \Lambda\\
{\dot H}^{a} & = \Lambda^{a} \ .
\end{align}
\end{subequations}
Write the resulting Hamiltonian density out explicitly, we have
\begin{align}
	\begin{split}
		\mathscr{H} &= \frac{\alpha}{\sqrt{g}}\left(P^{ab}P_{ab} - \frac{P^{2}}{2} - \frac{\alpha P \pi}{2} + \frac{\alpha^{2}\pi^{2}}{8} - \frac{\alpha^{2}}{2}\rho_{a}\rho^{a}\right)\\
		& -\alpha^{2}\pi H_{\perp} + \alpha^{2}\rho_{a}H^{a} + \alpha^{2}\Delta\Gamma^{a}_{~bc}g^{bc}\rho_{a} - \alpha \rho^{a}\partial_{a}\alpha - \alpha\sqrt{g}R\\
		& + P^{ab} \mathcal{L}_{\beta}g_{ab} + \pi \left[\beta^{c}\partial_{c}\alpha + \frac{\alpha}{{\bar \alpha}}\left({\dot {\bar \alpha}} - {\bar \beta}^{a}\partial_{a}{\bar \alpha}\right)\right]\\
		& + \rho_{a}\left[{\dot {\bar \beta}}^{a} + \frac{\Delta \beta^{a}}{{\bar \alpha}}\left({\dot {\bar \alpha}} - {\bar \beta}^{a}\partial_{a}{\bar \alpha}\right) + \beta^{b}{\bar D}_{b}\Delta \beta^{a} - \Delta \beta^{b} {\bar D}_{b}{\bar \beta}^{a}\right]\\
		& + \Omega \Lambda + \Omega_{a}\Lambda^{a}
	\end{split}
\end{align}
Consequently, we can write the action in Hamiltonian form as
\begin{equation}\label{non covariant action}
S[g, P, \alpha, \pi, \beta, \rho, H, \Omega, \Lambda ]\footnote{Indices of arguments are ommitted here for the purpose of a concise presentation. } = \int d^{4}x \left(P^{ab}{\dot g}_{ab} + \pi {\dot \alpha} + \rho_{a}{\dot \beta}^{a} + \Omega{\dot H_{\perp}} + \Omega_{a}{\dot H}^{a} - \mathscr{H}\right) \ .
\end{equation}
Hamilton's equations of motion can be derived from the action by applying the least action principle, i.e., $\delta S = 0$. Therefore, by varying the action with respect to the canonical variables, we have
\begin{subequations}\label{non covariant hamilton}
\begin{align}
{\dot g}_{ab} & = \frac{\partial \mathscr{H}}{\partial P^{ab}} = \mathcal{L}_{\beta}g_{ab} + \frac{2\alpha}{\sqrt{g}}P_{ab} - \frac{\alpha P}{\sqrt{g}}g_{ab} - \frac{\alpha^{2}\pi}{2\sqrt{g}}g_{ab}\\
\begin{split}
{\dot P}^{ab} & = -\frac{\partial \mathscr{H}}{\partial g_{ab}} = \mathcal{L}_{\beta}P^{ab} - \frac{2\alpha}{\sqrt{g}}P^{ac}P^{bd}g_{cd} + \frac{\alpha}{\sqrt{g}}PP^{ab} + \frac{\alpha^{2}\pi}{2\sqrt{g}
}P^{ab} - \frac{\alpha^{3}}{2\sqrt{g}}\rho^{a}\rho^{b}\\
& + \frac{\alpha}{2\sqrt{g}}P^{cd}P_{cd}g^{ab} - \frac{\alpha P^{2}}{4\sqrt{g}}g^{ab} - \frac{\alpha^{2}P\pi}{4\sqrt{g}}g^{ab} + \frac{\alpha^{2}\pi^{3}}{16\sqrt{g}}g^{ab} - \frac{\alpha^{3}}{4\sqrt{g}}\rho^{c}\rho_{c}g^{ab}\\
& + \alpha^{2}\rho_{e}\Delta \Gamma^{e}_{~cd}g^{ac}g^{bd} - \frac{1}{2}D_{c}\left(\rho^{c}\alpha^{2}\right)g^{ab} + D^{(a}\left(\rho^{b)}\alpha^{2}\right) - \frac{1}{2}\rho^{(a}D^{b)}\alpha^{2}\\
& - \alpha \sqrt{g}G^{ab} + \sqrt{g}D^{a}D^{b}\alpha - \sqrt{g}g^{ab}D_{c}D^{c}\alpha
\end{split}\\
{\dot \alpha} & = \frac{\partial \mathscr{H}}{\partial \pi} = \mathcal{L}_{\beta}\alpha  + \frac{\alpha}{{\bar \alpha}}\left({\dot {\bar \alpha}} - {\bar \beta}^{a}\partial_{a}{\bar \alpha}\right) - \frac{\alpha^{2}}{2\sqrt{g}}P + \frac{\alpha^{3}\pi}{4\sqrt{g}} - \alpha^{2}H_{\perp}\\
\begin{split}
{\dot \pi} & = - \frac{\partial \mathscr{H}}{\partial \alpha} = \mathcal{L}_{\beta}\pi - \frac{\pi}{{\bar \alpha}}\left({\dot {\bar \alpha}} - {\bar \beta}^{a}\partial_{a}{\bar \alpha}\right) + 2\alpha\pi H_{\perp} - 2\alpha \rho_{a}H^{a} - 2\alpha\Delta \Gamma^{a}_{~bc}g^{bc}\rho_{a}\\
&- \frac{1}{\sqrt{g}}P^{ab}P_{ab} + \frac{P^{2}}{2\sqrt{g}} + \frac{\alpha P \pi}{\sqrt{g}} - \frac{3\alpha^{2}\pi^{2}}{8\sqrt{g}} + \frac{3\alpha^{2}}{2\sqrt{g}}\rho_{a}\rho^{a} - \alpha \partial_{a}\rho^{a} + \sqrt{g}R
\end{split}\\
\begin{split}
{\dot \beta}^{a} & = \frac{\partial \mathscr{H}}{\partial \rho_{a}} = {\dot {\bar \beta}}^{a} + \left(\beta^{b}{\bar D}_{b}\Delta \beta^{a} - \Delta \beta^{b} {\bar D}_{b}{\bar \beta}^{a}\right) + \frac{\Delta \beta^{a}}{{\bar \alpha}}\left({\dot {\bar \alpha}} - {\bar \beta}^{a}\partial_{a}{\bar \alpha}\right) - \frac{\alpha^{3}}{\sqrt{g}}\rho^{a} + \alpha^{2}H^{a}\\
& + \alpha^{2}\Delta \Gamma^{a}_{~bc}g^{bc} - \alpha g^{ab}\partial_{b}\alpha
\end{split}\\
\begin{split}
{\dot \rho}_{a} & = -\frac{\partial \mathscr{H}}{\partial \beta^{a}} = - \rho_{b}{\bar D}_{a}\Delta\beta^{b} + \partial_{b}\left(\rho_{a}\beta^{b}\right) - \rho_{c}\beta^{b}{\bar \Gamma}^{c}_{~ab} + \rho_{b}{\bar D}_{a}{\bar \beta}^{b} - \frac{\rho_{a}}{{\bar \alpha}}\left({\dot {\bar \alpha}} - {\bar \beta}^{a}\partial_{a}{\bar \alpha}\right)\\
& - P^{bc}\partial_{a}g_{bc} + 2\partial_{b}\left(P^{bc}g_{ac}\right) - \pi \partial_{a}\alpha
\end{split}\\
{\dot H}_{\perp} & = \frac{\partial \mathscr{H}}{\partial \Omega } = \Lambda\\
{\dot \Omega} & = - \frac{\partial \mathscr{H}}{\partial H_{\perp}} = \alpha^{2}\pi\\
{\dot H}^{a} & = \frac{\partial \mathscr{H}}{\partial \Omega_{a}} = \Lambda^{a}\\
{\dot \Omega}_{a} & = - \frac{\partial \mathscr{H}}{\partial H^{a}} = -\alpha^{2}\rho_{a} \ .
\end{align}
\end{subequations}

One may notice that Eqs.~(\ref{non covariant hamilton}) are not manifestly covariant under foliation preserving transformation. This is a direct result from the fact that the action itself(\ref{non covariant action}) is not manifestly covariant. It is the presence of non-covariant time derivatives, such as $P^{ab}{\dot g}_{ab}$, $\pi{\dot \alpha}$, $\rho_{a}{\dot \beta}$, $\Omega{\dot H}_{\perp}$ and $\Omega_{a}{\dot H}^{a}$, that prevents the action from being manifestly covariant. However, a more careful observation reveals that, non-coincidentally, we can extract some terms from the hamiltonian $\mathscr{H}$ to combine with the ordinary time derivative terms and make them covariant time derivatives $D_{t}g_{ab}$, $D_{t}\alpha$, $D_{t}\beta^{a}$, as presented in Eqs.~(\ref{spatialmetriccovarianttimederivative}, \ref{lapsecovarianttimederivative}, \ref{shiftcovarianttimederivative}). Therefore, we start to modify the action presentation as following
\begin{equation}\label{modified action}
S[g, P, \alpha, \pi, \beta, \rho, H, \Omega, \Lambda ] = \int d^{4}x \left( P^{ab}D_{t}g_{ab} + \pi D_{t}\alpha + \rho_{a}D_{t}\beta^{a} + \Omega {\dot H}_{\perp} + \Omega_{a}{\dot H}^{a} - \tilde{\mathscr{H}}\right)
\end{equation}
where
\begin{align}
\begin{split}
\tilde{\mathscr{H}} & = \frac{\alpha}{\sqrt{g}}\left(P^{ab}P_{ab} - \frac{P^{2}}{2} - \frac{\alpha P \pi}{2} + \frac{\alpha^{2}\pi^{2}}{8} - \frac{\alpha^{2}}{2}\rho_{a}\rho^{a}\right)\\
& -\alpha^{2}\pi H_{\perp} + \alpha^{2}\rho_{a}H^{a} + \alpha^{2}\Delta\Gamma^{a}_{~bc}g^{bc}\rho_{a} - \alpha \rho^{a}\partial_{a}\alpha - \alpha\sqrt{g}R\\
& + \Omega \Lambda + \Omega_{a}\Lambda^{a}
\end{split}
\end{align}
is the modified Hamiltonian density. 

Note that in the modified action (\ref{modified action}), we still have two non-covariant terms $\Omega {\dot H}_{\perp}$ and $\Omega_{a}{\dot H}^{a}$ for which we are not able to find any complement terms in $\mathscr{H}$ to combine with. This problem can be solved by extending the Hamiltonian, which is discussed in detail in the following section. 
%%%%%%%%%%%%%%%%%%%%%%%%%%%%%%%%%%%%%%%%%%%%%%%%%
%%%%%%%%%%%%%%%%%%%%%%%%%%%%%%%%%%%%%%%%%%%%%%%%%
\section{Hamiltonian Extension}\label{extension}
Due to the Hamiltonian's constraints $\Omega = 0$ and $\Omega_{a} = 0$, we notice that the histories extremize the action, which contains $\Omega \Lambda$ and $\Omega_{a}\Lambda^{a}$, are invariant if we replace the multipliers by $\Lambda \rightarrow \Lambda + {\hat \Lambda}$ and $\Lambda^{a} \rightarrow \Lambda^{a} + {\hat \Lambda}^{a}$ while restraining ${\hat \Lambda}$ and ${\hat \Lambda}^{a}$ to be quasilinear functions of the canonical variables. What quasilinear means is that the principal parts of ${\hat \Lambda}$ and ${\hat \Lambda}^{a}$ are linear in the momenta $P^{ab}$, $\pi$, $\rho_{a}$, $\Omega$ and $\Omega_{a}$; and first spatial derivatives of the coordinates $\partial_{c}g_{ab}$, $\partial_{a}\alpha$, $\partial_{b}\beta^{a}$, $\partial_{a}H_{\perp}$ and $\partial_{b}H^{a}$ with coefficients depending on the coordinates. With these replacements, we can choose ${\hat \Lambda}$ and ${\hat \Lambda}^{a}$ appropriately so that they can combine with ${\dot H}_{\perp}$ and ${\dot H}^{a}$ in Eq.~(\ref{modified action}) to form $D_{t}H_{\perp}$ and $D_{t}H^{a}$ respectively, i.e., 
\begin{subequations}
\begin{align}
{\hat \Lambda} & \equiv \mathcal{L}_{\beta}H_{\perp}\\
{\hat \Lambda}^{a} & \equiv \mathcal{L}_{\beta}H^{a} \ .
\end{align}
\end{subequations}
From here we obtain a manifestly covariant format of the action as
\begin{equation}\label{covariant action}
\begin{split}
S[g, P, \alpha, \pi, \beta, \rho, H, \Omega, \Lambda ] = & \int d^{4}x ~~~~P^{ab}D_{t}g_{ab} + \pi D_{t}\alpha + \rho_{a}D_{t}\Delta \beta^{a} + \Omega D_{t}H_{\perp} + \Omega_{a}D_{t}H^{a}\\
& - \frac{\alpha}{\sqrt{g}}\left(P^{ab}P_{ab} - \frac{P^{2}}{2} - \frac{\alpha P \pi}{2} + \frac{\alpha^{2}\pi^{2}}{8} - \frac{\alpha^{2}}{2}\rho_{a}\rho^{a}\right)\\
& +\alpha^{2}\pi H_{\perp} - \alpha^{2}\rho_{a}H^{a} - \alpha^{2}\Delta\Gamma^{a}_{~bc}g^{bc}\rho_{a} + \alpha \rho^{a}\partial_{a}\alpha + \alpha\sqrt{g}R\\
& - \Omega\Lambda - \Omega_{a}\Lambda^{a} \ .
\end{split}
\end{equation} 

A careful reader would also notice that, besides making the action manifestly covariant, there is one extra change from Eq.~(\ref{modified action}) to Eq.~(\ref{covariant action}) as well. The term coupling with $\rho_{a}$ is $D_{t}\Delta \beta$ instead of $D_{t}\beta^{a}$ now. A superficial justification for this operation is that from Eq.~(\ref{shiftcovarianttimederivative}) we can tell that $D_{t}{\bar \beta}^{a} = 0$ and hence $D_{t}\Delta \beta^{a} = D_{t}\beta^{a}$. Prior to a deeper understanding of this operation, let's first take a look at our approach to vary the action to get a set of manifestly covariant Hamilton's equations. 

By manifestly covariant, what we want is that terms on the left hand side and right hand side of equations of motion are both covariant. Therefore, instead of a ordinary time derivative operator on the left hand side of the equations, we require a covariant time derivative operator. Hence, when varying the action, we want the terms with covariant time derivative have the same product rule as ordinary time derivatives, i.e., for any generic coordinate variable $q$ and its conjugate momentum $p$, we require that
\begin{equation}\label{productrule}
D_{t}( pq ) = pD_{t}q + qD_{t}p \ .
\end{equation}
We also require $D_{t}(pq)$ has a format of total derivative so that it can be treated as boundary terms while varying the action. This works as long as $pq$ is a weight 1 density under spatial diffeomorphism and a scalar density under time reparameterization, in which case we have
\begin{equation}
D_{t}(pq) = \partial_{t}(pq) - \mathcal{L}_{\beta}(pq) = \partial_{t}(pq) - \partial_{c}\left(\beta^{c}pq\right) \ .
\end{equation}
Here resides the reason why we want to replace $\rho_{a}D_{t}\beta^{a}$ with $\rho_{a}D_{t}\Delta \beta^{a}$, since $\rho_{a}\Delta \beta^{a}$ is a weight 1 density under spatial diffeomorphism and a scalar under time reparameterization while $\rho_{a}\beta^{a}$ is not. 

With explicit definition of $D_{t}q$, the product rule allows us to define $D_{t}p$ from Eq.~(\ref{productrule}) in the following way
\begin{subequations}
\begin{align}
D_{t}P^{ab} & \equiv {\dot P}^{ab} - \mathcal{L}_{\beta}P^{ab}\\
D_{t}\pi & \equiv {\dot \pi} - \mathcal{L}_{\beta}\pi + \frac{\pi}{{\bar \alpha}}\left({\dot {\bar \alpha}} - {\bar \beta}^{a}\partial_{a}{\bar \alpha}\right)\\
D_{t}\rho_{a} & \equiv {\dot \rho}_{a} - \partial_{b}\left(\beta^{b}\rho_{a}\right) + \frac{\rho_{a}}{{\bar \alpha}}\left({\dot {\bar \alpha}} - {\bar \beta}^{b}\partial_{b}{\bar \alpha}\right) + \rho_{c}\beta^{b}{\bar \Gamma}^{c}_{ab} - \rho_{b}{\bar D}_{a}{\bar \beta}^{b}\\
D_{t}\Omega & \equiv {\dot \Omega} - \mathcal{L}_{\beta}\Omega\\
D_{t}\Omega_{a} & \equiv {\dot \Omega}_{a} - \mathcal{L}_{\beta}\Omega_{a} \ .
\end{align}
\end{subequations}

With these two properties of covariant time derivative, we can vary the covariant action to obtain the explicitly covariant Hamilton's equations as below,
\begin{subequations}\label{covarianthamilton}
\begin{align}
D_{t}g_{ab} & = \frac{2\alpha}{\sqrt{g}}P_{ab} - \frac{\alpha P}{\sqrt{g}}g_{ab} - \frac{\alpha^{2}\pi}{2\sqrt{g}}g_{ab}\\
\begin{split}
D_{t} P^{ab} & = - \frac{2\alpha}{\sqrt{g}}P^{ac}P^{bd}g_{cd} + \frac{\alpha}{\sqrt{g}}PP^{ab} + \frac{\alpha^{2}\pi}{2\sqrt{g}
}P^{ab} - \frac{\alpha^{3}}{2\sqrt{g}}\rho^{a}\rho^{b}\\
& + \frac{\alpha}{2\sqrt{g}}P^{cd}P_{cd}g^{ab} - \frac{\alpha P^{2}}{4\sqrt{g}}g^{ab} - \frac{\alpha^{2}P\pi}{4\sqrt{g}}g^{ab} + \frac{\alpha^{2}\pi^{3}}{16\sqrt{g}}g^{ab} - \frac{\alpha^{3}}{4\sqrt{g}}\rho^{c}\rho_{c}g^{ab}\\
& + \alpha^{2}\rho_{e}\Delta \Gamma^{e}_{~cd}g^{ac}g^{bd} - \frac{1}{2}D_{c}\left(\rho^{c}\alpha^{2}\right)g^{ab} + D^{(a}\left(\rho^{b)}\alpha^{2}\right) - \frac{1}{2}\rho^{(a}D^{b)}\alpha^{2}\\
& - \alpha \sqrt{g}G^{ab} + \sqrt{g}D^{a}D^{b}\alpha - \sqrt{g}g^{ab}D_{c}D^{c}\alpha
\end{split}\\
D_{t}\alpha & = - \frac{\alpha^{2}}{2\sqrt{g}}P + \frac{\alpha^{3}\pi}{4\sqrt{g}} - \alpha^{2}H_{\perp}\\
\begin{split}
D_{t}\pi & = 2\alpha\pi H_{\perp} - 2\alpha \rho_{a}H^{a} - 2\alpha\Delta \Gamma^{a}_{~bc}g^{bc}\rho_{a} - \frac{1}{\sqrt{g}}P^{ab}P_{ab} + \frac{P^{2}}{2\sqrt{g}} + \frac{\alpha P \pi}{\sqrt{g}}\\
& - \frac{3\alpha^{2}\pi^{2}}{8\sqrt{g}} + \frac{3\alpha^{2}}{2\sqrt{g}}\rho_{a}\rho^{a} - \alpha \partial_{a}\rho^{a} + \sqrt{g}R
\end{split}\\
D_{t}\Delta \beta^{a} & = - \frac{\alpha^{3}}{\sqrt{g}}\rho^{a} + \alpha^{2}H^{a} + \alpha^{2}\Delta \Gamma^{a}_{~bc}g^{bc} - \alpha g^{ab}\partial_{b}\alpha\\
D_{t}\rho_{a} & = - \rho_{b}{\bar D}_{a}\Delta\beta^{b} - P^{bc}\partial_{a}g_{bc} + 2\partial_{b}\left(P^{bc}g_{ac}\right) - \pi \partial_{a}\alpha - \Omega \partial_{a}H_{\perp} - \Omega_{b}\partial_{a}H^{b} - \partial_{b}\left(H^{b} \Omega_{a}\right)\\
D_{t}H_{\perp} & = \Lambda\\
D_{t}\Omega & = \alpha^{2}\pi \label{omega}\\
D_{t}H^{a} & = \Lambda^{a}\\
D_{t}\Omega_{a} & = -\alpha^{2}\rho_{a} \label{omega_a} \ .
\end{align}
\end{subequations}
In order for the Hamiltonian formulation discussed above to be practical in numerical application, we first need to prove that Eqs.~(\ref{covarianthamilton}) is well-posed. The definition of well-posedness for quasilinear partial differential equation system is discussed in Sec.~\ref{wellposednessdefinition} in detail. A complete well-posedness analysis of Eqs.~(\ref{covarianthamilton}) is presented in Sec.~\ref{wellposednessanalysis}.
%%%%%%%%%%%%%%%%%%%%%
%%%%%%%%%%%%%%%%%%%%%
\section{Well-posedness}\label{wellposedness}
\subsection{Definition}\label{wellposednessdefinition}
The notion of ``well-posed'' is originated from Jacques Hadamard, a French mathematician. He argues that mathematical models of physical phenomena should have the properties that a unique solution exists and the solution's behavior changes continuously with the initial conditions. Wald gave his own yet similar definition of well-posedness in his book {\em General Relativity}\cite{Wald:GRbook}. Wald's definition states that in order for a numerical formulation to be well-posed, firstly small changes in initial data should produce only correspondingly small changes in the solution over any fixed compact region of spacetime; and secondly changes of initial data in a region should only be responsible for changes in the solution inside the causal future of this region. The first requirement is necessary since it is impossible to measure initial condition in infinite accuracy. The second property guarantees no signal could propagate faster than the speed of light and keep the numerical formulation under framework of relativity theory. 

An algebraic criterion is introduced and widely adopted to analyze well-posedness of a quasilinear partial differential equation system, namely {\em hyperbolicity}. An initial value problem with weakly hyperbolicity is regarded as ill-posed. Strongly hyperbolicity is necessary and sufficient for an initial value problem without boundaries in space to be well-posed. For initial value systems with space boundaries being well-posed, an even stronger notion of hyperbolicity, symmetric hyperbolicity, is required. Gundlach, Mart\' \i n-Garc\' \i a\cite{Gundlach:2005ta} and Brown\cite{Brown:2008cca} have presented practical definition of various types of hyperbolicity and prescriptions to prove them. Their theories to determine the hyperbolicity of a Hamiltonian system is briefly reviewed as following. 

Well-posedness guarantees small perturbation in initial data only leads to proportional deviations of the solution. Therefore in analyzing well-posedness we are primarily concerned with evolutions of high wave number modes in initial value perturbations. For large wave numbers, behavior of the Hamiltonian system depends on the coefficients of the highest weight terms in the Hamiltonian, namely the principal terms. The principal symbol $A$ is a square matrix obtained by identifying the principal terms in Hamilton's equations. According to Brown's prescription\cite{Brown:2008cca}, for a Hamiltonian system with generalized coordinates $q_{\mu}$ and momenta $p_{\mu}$ \footnote{Here the subscript $\mu$ is merely a counting index rather than tensor notation.}, the principal parts of ${\dot q}_{\mu}$ equations are the terms proportional to $p_{\mu}$ and $\partial_{a}q_{\mu}$; principal parts of ${\dot p}_{\mu}$ equations are the terms proportional to $\partial_{a}p_{\mu}$ and $\partial_{a}\partial_{b}q_{\mu}$. To construct the principal symbol $A$, one only needs to carry out the following substitution
\begin{subequations}\label{substitution}
\begin{align}
p_{\mu} & \rightarrow {\bar p}_{\mu}\\
\partial_{a}q_{\mu} &\rightarrow n_{a}{\bar q}_{\mu}\\
\partial_{a}p_{\mu} & \rightarrow n_{a}{\bar p}_{\mu}\\
\partial_{a}\partial_{b}p_{\mu} & \rightarrow n_{a}n_{b}{\bar p}_{\mu} \ ,
\end{align}
\end{subequations}
where ${\bar q}_{\mu}$ and ${\bar p}_{\mu}$ are Fourier mode amplitudes of corresponding perturbations for $q_{\mu}$ and $p_{\mu}$. The principal symbol $A$ is formed from the coefficients of ${\bar q}_{\mu}$ and ${\bar p}_{\mu}$. 

Solving eigen problem of the principal symbol $A$ reveals hyperbolicity of the PDE system in consideration. If all eigenvalues of $A$ are real but the eigenvectors are not complete, the system is regarded as weakly hyperbolic. By contrast, a system whose principal symbol $A$ has all real eigenvalues and a complete set of eigenvectors is said to be strongly hyperbolic. 

To further prove symmetric hyperbolicity, an energy term $\epsilon$ in quadratic form of the dynamic variables $p_{\mu}$, $q_{\mu}$ needs to be determined. Gundlach and Mart\' \i n-Garc\' \i a\cite{Gundlach:2005ta} pointed out that if $\epsilon$ is positive definite\footnote{$\epsilon = 0$ if and only if $q_{\mu} = 0$ and $p_{\mu} = 0$.} and the principal part of its time derivative can be written as the gradient of a vector $\phi^{a}$, i.e., $\partial_{t}\epsilon \cong \partial_{a}\phi^{a}$, the Hamiltonian system is considered to be symmetric hyperbolic. 

\subsection{Analysis}\label{wellposednessanalysis}
Following the recipe given above, we first extract the principal part of Eqs.~(\ref{covarianthamilton}) as
\begin{subequations}\label{principalhamilton}
\begin{align}
\partial_{\perp}g_{ab} &\cong 2g_{c(a}\partial_{b)}\beta^{c} + \frac{2\alpha}{\sqrt{g}}P_{ab} - \frac{\alpha P}{\sqrt{g}}g_{ab} - \frac{\alpha^{2}\pi}{2\sqrt{g}}g_{ab}\\
\begin{split}
\partial_{\perp}P^{ab} &\cong \frac{\alpha\sqrt{g}}{2}g^{ac}g^{bd}g^{ef}\left(\partial_{c}\partial_{d}g_{ef} + \partial_{e}\partial_{f}g_{cd} - 2\partial_{e}\partial_{(c}g_{d)f}\right)\\
& + \frac{\alpha\sqrt{g}}{2}g^{ab}g^{cd}g^{ef}\left(\partial_{c}\partial_{e}g_{df} - \partial_{c}\partial_{d}g_{ef}\right)\\
& + \sqrt{g}\left(g^{ac}g^{bd} - g^{ab}g^{cd}\right)\partial_{c}\partial_{d}\alpha\\
& + \alpha^{2}\left(g^{c(a}g^{b)d} - \frac{1}{2}g^{ab}g^{cd}\right)\partial_{c}\rho_{d}
\end{split}\\
\partial_{\perp}\alpha & \cong -\frac{\alpha^{2}}{2\sqrt{g}}P + \frac{\alpha^{3}}{4\sqrt{g}}\pi\\
\partial_{\perp}\pi & \cong -\alpha g^{ab}\partial_{a}\rho_{b} + \sqrt{g}\left(g^{ac}g^{bd} - g^{ab}g^{cd}\right)\partial_{a}\partial_{b}g_{cd}\\
\partial_{\perp}\beta^{a} & \cong -\alpha g^{ab}\partial_{b}\alpha - \frac{\alpha^{3}}{\sqrt{g}}g^{ab}\rho_{b} + \alpha^{2}\left(g^{ac}g^{bd} - \frac{1}{2}g^{ab}g^{cd}\right)\partial_{b}g_{cd}\\
\partial_{\perp}\rho_{a} & \cong 2g_{ab}\partial_{c}P^{bc} - H^{b}\partial_{b}\Omega_{a}\\
\partial_{\perp}H_{\perp} & \cong 0\\
\partial_{\perp}\Omega & \cong 0\\
\partial_{\perp}H^{a} & \cong 0\\
\partial_{\perp}\Omega_{a} & \cong 0 \ .
\end{align}
\end{subequations}
The $\cong$ symbol denotes equality up to non principal terms. These equations are expressed in terms of the operator $\partial_{\perp} \equiv \partial_{t} - \beta^{a}\partial_{a}$ so that the characteristic speeds are defined with respect to observers who are at rest in the spacelike slices. 

The next step is to construct the principal symbol $A$ by making the replacement described in Eqs.~(\ref{substitution}) and identifying coefficients on the right-hand sides of Eqs.~(\ref{principalhamilton}). We also divide these coefficients by a factor of $\alpha$ so that the characteristic speeds can be expressed in terms of proper time rather than coordinate time. The resulting eigenvalue problem is
\begin{subequations}\label{eigensystem}
\begin{align}
\mu {\bar g}_{ab} & = \frac{2}{\alpha}n_{(a}{\bar \beta}_{b)} + \frac{2}{\sqrt{g}}{\bar P}_{ab} - \frac{g_{ab}}{\sqrt{g}}\left({\bar P}_{nn} + {\bar P}_{AB}\delta^{AB}\right) - \frac{\alpha g_{ab}}{2\sqrt{g}}{\bar \pi}\\
\begin{split}
\mu {\bar P}_{ab} & = \frac{\sqrt{g}}{2}n_{a}n_{b}\left({\bar g}_{nn} + {\bar g}_{AB}\delta^{AB}\right) + \frac{\sqrt{g}}{2}{\bar g}_{ab} - \sqrt{g} n_{(a}{\bar g}_{b)n} - \frac{\sqrt{g}}{2}g_{ab}{\bar g}_{AB}\delta^{AB}\\
& + \frac{\sqrt{g}}{\alpha}n_{a}n_{b}{\bar \alpha} - \frac{\sqrt{g}}{\alpha}g_{ab}{\bar \alpha} + \alpha n_{(a}{\bar \rho}_{b)} - \frac{1}{2}\alpha g_{ab}{\bar \rho}_{n}
\end{split}\\
\mu {\bar \alpha} & = -\frac{\alpha}{2\sqrt{g}}\left({\bar P}_{nn} + {\bar P}_{AB}\delta^{AB}\right) + \frac{\alpha^{2}}{4\sqrt{g}}{\bar \pi}\\
\mu {\bar \beta}_{a} & = -n_{a}{\bar \alpha} - \frac{\alpha^{2}}{\sqrt{g}}{\bar \rho}_{a} + {\bar \alpha}{\bar g}_{nn} - \frac{1}{2}\alpha n_{a}\left({\bar g}_{nn} + {\bar g}_{AB}\delta^{AB}\right)\\
\mu {\bar \rho}_{a} & = \frac{2}{\alpha}{\bar P}_{an} - \frac{\left(H \cdot n \right)}{\alpha}{\bar \Omega}_{a}\\
\mu {\bar H}_{\perp} & = 0\\
\mu {\bar \Omega} & = 0\\
\mu {\bar H}_{a} & = 0\\
\mu {\bar \Omega}_{a} & = 0 \ .
\end{align}
\end{subequations}
Eqs.~(\ref{eigensystem}) can be interpreted as an eigensystem $\mu v = A v$ with eigenvalue $\mu$, where the eigenvector $v = [{\bar g}_{ab}, {\bar P}_{ab}, {\bar \alpha}, {\bar \pi}, {\bar \beta}_{a}, {\bar \rho}_{a}, {\bar H}_{\perp}, {\bar \Omega}, {\bar H}_{a}, {\bar \Omega}_{a}]^{T}$ and $A$ is the principal symbol we need. 

In Eqs.~(\ref{eigensystem}), a subscript $n$ denotes contraction with $n^{a}$. We introduce an orthonormal diad $e^{a}_{A}$ with $A = 1, 2$ in the subspace orthogonal to $n_{a}$, which means $n_{a}e^{a}_{A} = 0$ and $e^{a}_{A}g^{ab}e^{b}_{B} = \delta^{AB}$. A subscript A denotes contraction with $e^{a}_{A}$.

We can further split the eigensystem (\ref{eigensystem}) into scalar, vector and trace-free tensor blocks by contracting the equations with $n^{a}$ and/or $e^{a}_{A}$. Since these sectors can be interpreted as diagonal blocks of the principal symbol $A$, each block can be studied individually to gain some insight of the system's hyperbolicity. 

First of all, the scalar block is
\begin{subequations}
\begin{align}
\mu {\bar g}_{nn} & = \frac{2}{\alpha}{\bar \beta}_{n} + \frac{1}{\sqrt{g}}{\bar P}_{nn} - \frac{1}{\sqrt{g}}{\bar P}_{AB}\delta^{AB} - \frac{\alpha}{2\sqrt{g}}{\bar \pi}\\
\mu {\bar g}_{AB}\delta^{AB} & = - \frac{2}{\sqrt{g}}{\bar P}_{nn} - \frac{\alpha}{\sqrt{g}}{\bar \pi}\\
\mu {\bar P}_{nn} & = \frac{1}{2}\alpha{\bar \rho}_{n}\\
\mu {\bar P}_{AB}\delta^{AB} & = -\frac{\sqrt{g}}{2}{\bar g}_{AB}\delta^{AB} - \frac{2}{\alpha}\sqrt{g} {\bar \alpha} - \alpha {\bar \rho}_{n}\\
\mu {\bar \alpha} & = - \frac{\alpha}{2\sqrt{g}}\left({\bar P}_{nn} + {\bar P}_{AB}\delta^{AB}\right) + \frac{\alpha^{2}}{4\sqrt{g}}{\bar \pi}\\
\mu {\bar \pi} & = -\frac{\sqrt{g}}{\alpha}{\bar g}_{AB}\delta^{AB} - {\bar \rho}_{n}\\
\mu {\bar \beta}_{n} & = - {\bar \alpha} - \frac{\alpha^{2}}{\sqrt{g}}{\bar \rho}_{n} + \frac{\alpha}{2}{\bar g}_{nn} - \frac{\alpha}{2}{\bar g}_{AB}\delta^{AB}\\
\mu {\bar \rho}_{n} & = \frac{2}{\alpha}{\bar P}_{nn} -  \frac{\left(H \cdot n \right)}{\alpha}{\bar \Omega}_{n}\\
\mu {\bar H}_{\perp} & = 0\\
\mu {\bar \Omega} & = 0\\
\mu {\bar H}_{n} & = 0\\
\mu {\bar \Omega}_{n} & = 0 \ .
\end{align}
\end{subequations}
The vector sector is
\begin{subequations}
\begin{align}
\mu {\bar g}_{nA} & = \frac{2}{\sqrt{g}}{\bar P}_{nA} + \frac{1}{\alpha}{\bar \beta}_{A}\\
\mu {\bar P}_{nA} & = \frac{\alpha}{2}{\bar \rho}_{A}\\
\mu {\bar \beta}_{A} & = \alpha {\bar g}_{nA} - \frac{\alpha^{2}}{\sqrt{g}}{\bar \rho}_{A}\\
\mu {\bar \rho}_{A} & = \frac{2}{\alpha}{\bar P}_{nA} - \frac{\left(H \cdot n \right)}{\alpha}{\bar \Omega}_{A}\\
\mu {\bar H}_{A} & = 0\\
\mu {\bar \Omega}_{A} & = 0 \ .
\end{align}
\end{subequations}
At last, the trace-free tensor sector is
\begin{subequations}
\begin{align}
\mu {\bar g}^{tf}_{AB} & = \frac{2}{\sqrt{g}}{\bar P}^{tf}_{AB}\\
\mu {\bar P}^{tf}_{AB} & = \frac{\sqrt{g}}{2}{\bar g}^{tf}_{AB} \ .
\end{align}
\end{subequations}
Solving the three sub-eigenproblems reveals that eigenvalues for the scalar sector are $\pm 1$ and 0, each with multiplicity four; eigenvalues for the vector sector are $\pm 1$ with multiplicity two and 0 with multiplicity two as well; eigenvalues for the trace-free tensor sector are $\pm 1$; eigenvectors for all the three sectors are complete, which draws the conclusion that the system is strongly hyperbolic. 

To further prove this system is symmetric hyperbolic, we apply Gundlach and Matrtin-Garcia\cite{Gundlach:2005ta}'s definition of symmetric hyperbolicity for quasilinear systems of partial differential equations with first-order time and second-order space derivatives and follow the analysis described in Ref.~\cite{Brown:2011qg}. To begin with, we introduce the spatial derivates of the 3 + 1 coordinate variables $g_{ab}, \alpha, \beta^{a}, H_{\perp}$ and $H^{a}$ as a new set of variables
\begin{subequations}
\begin{align}
g_{cab} & \equiv \partial_{c}g_{ab}\\
\alpha_{a} & \equiv \partial_{a}\alpha\\
\beta_{ab} & \equiv g_{ac}\partial_{b}\beta^{c}\\
H_{\perp a} & \equiv \partial_{a}H_{\perp}\\
H_{ab} & \equiv g_{ac}\partial_{b}H^{c} \ .
\end{align}
\end{subequations}
Then we compute the equations of motion for this new set of variables by differentiating Eqs.~(\ref{covarianthamilton}). Combine the resulting equations of motion with Hamilton's equations for the momenta, we obtain a new system as following(up to the principal terms)
\begin{subequations}\label{symmetric hamilton}
\begin{align}
\partial_{\perp}g_{cab} & \cong 2\partial_{c}\beta_{(ab)} + \frac{2\alpha}{\sqrt{g}}\partial_{c}P_{ab} - \frac{\alpha}{\sqrt{g}}g_{ab}g^{de}\partial_{c}P_{de} - \frac{\alpha^{2}}{2\sqrt{g}}g_{ab}\partial_{c}\pi\\
\begin{split}
\partial_{\perp}P_{ab} & \cong \frac{\alpha\sqrt{g}}{2}g^{cd}\left(\partial_{a}g_{bcd} + \partial_{c}g_{dab} - 2\partial_{c}g_{(ab)d}\right)\\
& + \frac{\alpha \sqrt{g}}{2}g_{ab}g^{cd}g^{ef}\left(\partial_{c}g_{edf} - \partial_{c}g_{def}\right)\\
& + \sqrt{g}\left(\partial_{a}\alpha_{b} - g_{ab}g^{cd}\partial_{c}\alpha_{d}\right) + \alpha^{2}\left(\partial_{(a}\rho_{b)} - \frac{1}{2}g_{ab}g^{cd}\partial_{c}\rho_{d}\right)
\end{split}\\
\partial_{\perp}\alpha_{a} & \cong -\frac{\alpha^{2}}{2\sqrt{g}}g^{cd}\partial_{a}P_{cd} + \frac{\alpha^{3}}{4\sqrt{g}}\partial_{a}\pi\\
\partial_{\perp}\pi & \cong - \alpha g^{ab}\partial_{a}\rho_{b} + \sqrt{g}\left(g^{ac}g^{bd} - g^{ab}g^{cd}\right)\partial_{a}g_{bcd}\\
\partial_{\perp}\beta_{ab} & \cong -\alpha \partial_{a}\alpha_{b} - \frac{\alpha^{3}}{\sqrt{g}}\partial_{a}\rho_{b} + \alpha^{2}\left(g^{cd}\partial_{a}g_{cbd} - \frac{1}{2}g^{cd}\partial_{a}g_{bcd}\right)\\
\partial_{\perp}\rho_{a} & \cong 2g^{bc}\partial_{c}P_{ab} - H^{b}\partial_{b}\Omega_{a}\\
\partial_{\perp}H_{\perp a} & \cong 0\\
\partial_{\perp}\Omega & \cong 0\\
\partial_{\perp}H_{ab} & \cong 0\\
\partial_{\perp}\Omega_{a} & \cong 0.
\end{align}
\end{subequations}
The quadratic energy term is defined as follows
\tiny
\begin{align}
\begin{split}
\epsilon = & M^{abef}\left[g^{cd}\left(\frac{g_{cab}}{2} - \frac{\alpha_{c}}{\alpha}g_{ab}\right)\left(\frac{g_{def}}{2} - \frac{\alpha_{d}}{\alpha}g_{ef}\right) + \left(\frac{P_{ab}}{\sqrt{g}} + \frac{\beta_{(ab)}}{\alpha} - \frac{\alpha \pi g_{ab}}{2\sqrt{g}}\right)\left(\frac{P_{ef}}{\sqrt{g}} + \frac{\beta_{(ef)}}{\alpha} - \frac{\alpha \pi g_{ef}}{2\sqrt{g}}\right) + H_{ab}H_{ef}\right]\\
& + M^{ab}\left[\frac{g^{cd}}{\alpha^{2}}\left(\beta_{(ac)} + \frac{\alpha}{\sqrt{g}}H_{(a}\Omega_{c)}\right)\left(\beta_{(bd)} + \frac{\alpha}{\sqrt{g}}H_{(b}\Omega_{d)}\right) + \left(\frac{\alpha \rho_{a}}{\sqrt{g}} - \Gamma_{acd}g^{cd} + \frac{\alpha_{a}}{\alpha}\right) \left(\frac{\alpha \rho_{b}}{\sqrt{g}} - \Gamma_{bcd}g^{cd} + \frac{\alpha_{b}}{\alpha}\right) + H_{\perp a}H_{\perp b} + \Omega_{a}\Omega_{b}\right]\\
& + M\left[g^{ab}g^{cd}\left(\beta_{ab} - \frac{2\alpha}{\sqrt{g}}P_{ab}\right)\left(\beta_{cd} - \frac{2\alpha}{\sqrt{g}}P_{cd}\right) + g^{cd}\left(\frac{\alpha g^{ab}}{2}g_{cab} + 3\alpha_{c}\right)\left(\frac{\alpha g^{ab}}{2}g_{dab} + 3\alpha_{d}\right) + \Omega^{2}\right]
\end{split}
\end{align}
\normalsize
where the tensors $M^{abef}$, $M^{ab}$ and $M$ are positive definite. Using Eqs.~(\ref{symmetric hamilton}) shows that the time derivative of $\epsilon$ has a principal part that can be written as the gradient of a vector $\phi^{a}$; that is, ${\dot \epsilon} \cong \partial_{a}\phi^{a}$. This leads to the conclusion that $\epsilon$ is a quadratic, positive-definite energy density with flux $\phi^{a}$ and hence the PDE system (\ref{covarianthamilton}) is symmetric hyperbolic. 

Since the generalized harmonic formulation is well known to be symmetric hyperbolic, we hereby conclude that the Hamiltonian approach to GH formulation developed in this paper conserves this merit and is practical to numerical applications. 



\chapter{Summary}
This paper is set to construct a Hamiltonian formulation for generalized harmonic interpretation of general relativity. As an en route investigation, this paper defines a missing notion of the 3 + 1 decomposition, the covariant time derivative $D_{t}$. As an analogy to covariant spatial derivative, the operator $D_{t}$ defines a covariant version of the regular time derivative $\partial_{t}$. Since the expression of $D_{t}$ is transformation dependent, in this paper it is defined under a general class of coordinate transformation, the foliation preserving transformation, which consists of a time reparameterization and a time dependent spatial diffeomorphism. General covariance was one of the postulates Einstein posted when he was constructing his theory of relativity and has always been a major part of the elegance credited to it, therefore we take advantage of the constructed covariant time derivative to recover general covariance of moving puncture gauge condition in BSSN formulation and damped-wave gauge condition in generalized harmonic equations. 

Later in this paper, a Hamiltonian density is derived directly from the predefined action of general harmonic formulation. It is then recognized as not covariant under our foliation preserving coordinate transformation. Understanding of the covariant time derivatives under foliation preserving coordinate transformation provides us with insights about how to extend the Hamiltonian density to achieve a set of covariant Hamilton's equations. This system of covariant evolution equations is later proved to be symmetric hyperbolic by following a series of well-established prescriptions. Hence we conclude that the Hamiltonian approach to generalized harmonic formulation preserves its well-posedness. 

The symmetric hyperbolicity of our newly developed Hamiltonian system guarantees its empiricalness in numerical applications. Although no numerical experiments are included in the scope of this work, we urge a numerical simulation to be build upon the Hamilton's equations listed in this paper to test its practical performance. With the covariant time derivatives and gauge conditions defined in this paper, we hope to improve the robustness, effectiveness and efficiency of existing numerical relativity formulations when multiple coordinate systems are required. Furthermore, the Hamiltonian formulation of generalized harmonic formalism presented in this paper introduces an alternate algorithm for simulating Einstein field equations. Meanwhile we'd also like to present a different perspective toward numerical relativity and hope new insights and theories can be sparkled by our work. 

%%---------------------------------------------------------------------------%%
%%  Bibliography 

%%  You can use the bibitem list.
%\bibliographystyle{unsrt}
%\begin{thebibliography}{99}
%\bibitem{cb02}
%Casella, G. and Berger, R.L. (2002)
%\newblock {\it Statistical Inference, Second Edition.}
%Duxbury Press, Belmont, CA.
%
%\bibitem{t06}
%Tsiatis, A.A. (2006)
%\newblock {\it Semiparametric Theory and Missing Data.}
%Springer, New York.
%
%\end{thebibliography}

%% or use BibTeX
\bibliography{meng-thesis}{}
\bibliographystyle{plain}

%%---------------------------------------------------------------------------%%
% Appendices
%\appendix



%%---------------------------------------------------------------------------%%
\backmatter


\end{document}
