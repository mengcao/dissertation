\chapter{General Covariance}\label{generalcovariance}
\section{Overview}
The Einstein field equations take the same 
form in any spacetime coordinate system. We refer to this property as {\em general covariance}.\footnote{The definition of 
general covariance is not entirely uniform throughout the literature \cite{Weinberg:GRbook,Misner:1974qy,Wald:GRbook,Lambourne:GRbook}. Most authors 
define general covariance, as we do, to include the concept of form invariance under changes of coordinates.} 
With the Einstein equations written as $G_{\mu\nu} = 8\pi T_{\mu\nu}$, and given the transformation properties 
of the individual terms $G_{\mu\nu}$ and $T_{\mu\nu}$, general covariance is easy to show. To do so, we first observe that 
the Einstein tensor $G_{\mu\nu}$ and the stress--energy tensor $T_{\mu\nu}$ transform as type $0\choose 2$ tensors under changes 
of spacetime coordinates. Thus, under the  
transformation $x^{\mu'} = x^{\mu'}(x^\mu)$, we have
\begin{equation}
	G_{\mu\nu} = \frac{\partial x^{\mu'}}{\partial x^{\mu}} \frac{\partial x^{\nu'}}{\partial x^{\nu}} G_{\mu'\nu'} 
\end{equation}
with a corresponding expression for $T_{\mu\nu}$. By applying the coordinate transformation 
to the Einstein equations we find that the factors 
$(\partial x^{\mu'}/\partial x^{\mu}) (\partial x^{\nu'}/\partial x^{\nu})$
cancel, leaving $G_{\mu'\nu'} = 8\pi T_{\mu'\nu'}$. 

Equations like $G_{\mu\nu} = 8\pi T_{\mu\nu}$, whose general covariance is trivial to confirm by virtue 
of the transformation properties of the individual terms, will be referred to as {\em manifestly covariant}. 

In the field of numerical relativity, Einstein's theory is interpreted as an initial value problem. As Einstein field equations being written in $3 + 1$ format, rather than {\em spacetime} coordinate transformations, we focus our attention on achieving general covariance under another class of coordinate transformations. This new family of coordinate transformations breaks time and space symmetry in the same sense of $3 + 1$ decomposition and general covariance of various numerical formulations under this series of coordinate transformations is discussed in the next section. 
%%%%%%%%%%%%%%%%%%%%%%%%%%%%%%%%%%%%%%%%%%%%%%%%%%%%%%%
\section{General Covariance under Foliation Preserving Transformations}\label{generalcovariancefoliationpreservingtransformation}
The ADM equations used by Smarr and York \cite{Smarr:York,Smarr:1977uf} are
\begin{subequations}\label{ADMeqns}
\begin{eqnarray}
	\partial_t g_{ab} - {\cal L}_\beta g_{ab} & = & -2\alpha K_{ab} \ , \\
	\partial_t K_{ab} - {\cal L}_\beta K_{ab} & = & \alpha K K_{ab} - 2\alpha K_{ac}K^c_b 
		+ \alpha R_{ab} - D_a D_b\alpha \ .
\end{eqnarray}
\end{subequations}
Matter fields have been set to zero for simplicity.  These equations are expressed in terms of the 
spatial metric $g_{ab}$, extrinsic curvature $K_{ab}$, lapse function $\alpha$ and shift vector $\beta^a$. 
Since the ADM equations hold in any spacetime coordinate system, they are generally covariant. They are not, however, 
manifestly covariant. This is because the ADM equations are written in terms  
fields $g_{ab}$,  $K_{ab}$, $\alpha$ and $\beta^a$, that do not transform as tensors under 
spacetime coordinate transformations. 

Although the ADM equations are not manifestly covariant under changes of {\em spacetime} coordinates, they are 
manifestly covariant under changes of {\em spatial} coordinates. This is because each term in Eqs.~(\ref{ADMeqns})
transforms as a type $0\choose 2$ tensor under spatial coordinate transformations. In fact, we will show that the ADM  
equations are manifestly covariant under a wider class of transformations that preserve the foliation of spacetime 
into spacelike hypersurfaces. These transformations are defined by
\begin{subequations}\label{foliationpreservingtransf}
\begin{eqnarray}
t' & = & t'(t) \ ,\\
x^{a'} & = & x^{a'}(t,x^{a}) \ .
\end{eqnarray}
\end{subequations}
They consist of time reparametrizations (\ref{foliationpreservingtransf}a) and {\em time--dependent} spatial coordinate transformations 
(\ref{foliationpreservingtransf}b). We will often refer to Eqs.~(\ref{foliationpreservingtransf}) as ``foliation preserving" transformations. 

The ADM equations provide the foundation for interpreting Einstein's theory as an initial value problem, 
and for solving the Einstein equations numerically. 
The lack of manifest spacetime covariance is not a problem in this context. In fact, in writing the Einstein equations in 3+1 
form, our point of view shifts away from 
spacetime fields to time--dependent fields on space. The manifest covariance of the ADM equations under foliation preserving 
transformations both supports and motivates this point of view. 

Let us consider more carefully the concept of general covariance in the $3+1$ context. 
In general relativity, coordinate systems themselves have little meaning apart from certain 
global or topological restrictions that they might impose. For example, if a coordinate $x^3$ ranges from $0$ to $2\pi$ 
and we impose periodic boundary conditions on the fields, then the $x^3$ coordinate lines are topologically circles. 
Although we often use names like ``Cartesian coordinates" or ``spherical coordinates", these names do not tell us anything 
about the geometry. They 
are merely a shorthand way of communicating information about the ranges of the coordinates and the topology of the underlying manifold. 

What does have a definite meaning is the comparison of fields using two different coordinate systems. For example, let 
$t,x^{a}$ and $t',x^{a'}$ denote  
two different coordinate systems, related as in Eq.~(\ref{foliationpreservingtransf}). We can carry out two separate numerical simulations, 
evolving the fields in the unprimed and primed coordinate systems, and compare the results to one another.  

How is this comparison done? 
First we prepare initial data with coordinates labeled $x^a$, letting $t_i$ denote the initial time. This is the unprimed initial data. We evolve this data to a final 
time $t_f$ using some set of partial differential equations. Next, we use 
the spatial transformation $x^{a'} = x^{a'}(t_i,x^a)$ and the transformation rules for the fields to transform the unprimed initial data to the primed 
system. This transformed data is the initial data for the evolution in primed coordinates, starting at the initial time $t'_i = t'(t_i)$. 
The differential equations used to evolve the primed data are obtained from the ``unprimed differential equations" using the transformations (\ref{foliationpreservingtransf}) and 
the transformation rules for the fields. If the equations are generally covariant under these transformations, then the primed equations have the same form 
as the unprimed equations. Using the primed evolution equations, we then evolve the primed data to the final 
time $t'_f = t'(t_f)$. The final data in primed coordinates will be related to the final data in unprimed coordinates by the coordinate transformation 
$x^{a'} = x^{a'}(t_f,x^a)$. 

The ADM equations are covariant, so they take the same form in the primed coordinates as in the unprimed coordinates. However, it is now 
well recognized that the ADM equations 
are not suited for numerical applications. Current numerical codes use one of several alterative systems, such as the 
Baumgarte--Shapiro--Shibata--Nakamura (BSSN) equations \cite{Shibata:1995we,Baumgarte:1998te} or the generalized harmonic (GH) 
equations \cite{Friedrich:1985,Garfinkle:2001ni}.
The BSSN and GH equations are generally covariant under foliation preserving transformations but, as they are usually written, 
they not manifestly covariant. The lack of manifest covariance stems from the use of non--tensorial fields as independent variables. 

Each of the systems, BSSN and GH, must be supplemented with a set of equations that fix the gauge freedom. 
The gauge conditions in current use are not generally covariant. In particular, the ``moving puncture gauge", commonly used with BSSN, is not covariant. 
The moving puncture gauge consists of 1+log slicing \cite{Bona:1994dr} and the
gamma--driver shift condition \cite{Alcubierre:2002kk}.  Likewise, the gauge conditions used 
with the GH equations \cite{Pretorius:2006tp,Szilagyi:2009qz, Lindblom:2009tu} are not covariant.  

For a non--covariant gauge condition, the form of the condition depends on the choice of coordinate system. 
In practice, one writes the gauge  in a form intended for use with ``Cartesian coordinates". This prescription is not  well--defined 
since Cartesian coordinates are not unique. Nevertheless, one chooses some set of initial data expressed in coordinates that are ``topologically Cartesian"  
and then writes the gauge equations in these coordinates.  One can then use the transformation rules for the fields to write the gauge conditions in terms of 
other coordinate systems. As we will show, these calculations can be tedious. 

The alternative discussed in this paper is to rewrite the gauge conditions in a form that is generally covariant under foliation preserving 
transformations (\ref{foliationpreservingtransf}). The gauge equations then take the same form in any coordinate system, and there is no need derive a new form 
for each coordinate system that one might consider. To carry out the analysis, we first derive the transformation rules for the ADM variables $\alpha$, $\beta^a$, 
$g_{ab}$, $K_{ab}$, and their time derivatives, under the foliation preserving transformations (\ref{foliationpreservingtransf}). Next, we introduce a 
set of background fields that allows us to group terms into combinations that transform as tensors. The gauge conditions for the BSSN and GH systems are 
then written in manifestly covariant form using these tensor combinations. 

In Ref.~\cite{Brown:2009dd} the BSSN equations and the moving puncture gauge were written in a form that is manifestly covariant under spatial coordinate transformations. 
Likewise, in Ref.~\cite{Brown:2011qg}, the GH equations were written in 3+1 form with manifest covariance 
under spatial coordinate transformations.  Here we write both systems, and their associated gauge conditions, in a form 
that is manifestly covariant under the more general foliation preserving transformations. 

%%%%%%%%%%%%%%%%%%%%%%%%%%%%%%%%%%%%%%%%%%%%%%%%%%%%%%%%%
\section{Transformation Rules}\label{transform}
With foliation preserving coordinate transformation being defined in the previous section, we start to investigate transformation rules of various important physical variables under this coordinate transformation, such as $g_{ab}$, $\alpha$ and $\beta^{a}$. More specifically, we try to determine which physical variables transform tensorially under foliation preserving coordinate transformation. For an arbitrary tensor density $T^{a_{1}...a_{n}}_{~~~~~~~~b_{1}...b_{m}}$, if it transforms as
\begin{equation}
	T^{a_{1}'...a_{n}'}_{~~~~~~~~b_{1}'...b_{m}'} = T^{a_{1}...a_{n}}_{~~~~~~~~b_{1}...b_{n}}\frac{\partial x^{a_{1}'}}{\partial x^{a_{1}}}...\frac{\partial x^{a_{n}'}}{\partial x^{a_{n}}}\frac{\partial x^{b_{1}}}{\partial x^{b_{1'}}}...\frac{\partial x^{b_{m}}}{\partial x^{b_{m'}}}\left|\frac{\partial t}{\partial t'}\right|^{i}\left|\frac{\partial x}{\partial x'}\right|^{j} \ ,
\end{equation}
it is regarded as ``transform tensorially" in this context. $\left|\frac{\partial t}{\partial t'}\right|$ and $\left|\frac{\partial x}{\partial x'}\right|$ are the Jacobian matrix determinant of time reparemeterization and spatial diffeomorphism, respectively. We refer this class of variables as a weight $i$ density under time reparameterization and a weight $j$ type $n \choose m$ tensor density under time dependent spatial diffeomorphism. If one quantity turns out to be a non-tensor, it will be ``tensorized" by combining with background metric terms, other physical variables and their derivatives. 

As discussed in Ref.~\cite{Brown:2010rya, Brown:2011qg}, background metric is introduced to achieve general covariance of non-tensor physical variables. An analogy of this tensorize action the use of background connection in spacetime coordinate transformations. It is well recognized that the four dimensional physical connection $\Gamma^{\alpha}_{~\mu\nu}$ is not a tensor. However, the difference between the physical connection and background connection $\Delta \Gamma^{\alpha}_{~\mu\nu} \equiv \Gamma^{\alpha}_{~\mu\nu} - {\bar \Gamma}^{\alpha}_{~\mu\nu}$ is a type $1 \choose 2$ tensor. The fact that background metric terms are often neglected from practical numerical equations can be justified by choosing the background metric to be flat. 

We start by listing several important identities revealed by the transformation rule of delta symbol. The inverse transformation of Eqs.~(\ref{foliationpreservingtransf}) is  
\begin{subequations}
\begin{eqnarray}
t & = & t(t')\\
x^{a} & = & x^{a}(t',x^{a'}) \ .
\end{eqnarray}
\end{subequations}
Use the chain rule of partial differentiation to inspect the delta function for this transformation, we have
\begin{equation}\label{chainrule}
\delta^{a}_{b} = \frac{\partial x^{a}}{\partial x^{b}} = \frac{\partial x^{a}}{\partial x^{\mu'}}\frac{\partial x^{\mu'}}{\partial x^{b}} = \frac{\partial x^{a}}{\partial t'}\frac{\partial t'}{\partial x^{b}} + \frac{\partial x^{a}}{\partial x^{a'}}\frac{\partial x^{a'}}{\partial x^{b}} \ .
\end{equation}
Since $t'\left(t\right)$ is independent of $x^{a}$, the term with $\partial t'/\partial x^{b}$ vanishes in Eq.~\ref{chainrule}. Hence we obtain an important identity of foliation preserving coordinate transformation
\begin{equation}\label{delta}
\frac{\partial x^{a}}{\partial x^{a'}}\frac{\partial x^{a'}}{\partial x^{b}} = \delta^{a}_{b}
\end{equation}
and equivalently
\begin{equation}\label{deltainverse}
\frac{\partial x^{a'}}{\partial x^{a}}\frac{\partial x^{a}}{\partial x^{b'}} = \delta^{a'}_{b'} \ .
\end{equation}
Although they are not directly referred to, Eqs.~(\ref{delta}, \ref{deltainverse}) are extensively used in the following derivations. We list them here for readers who want to reiterate our work. 

To achieve transformation rules of the spatial metric $g_{ab}$, lapse function $\alpha$ and shift vector $\beta^{a}$, we need to start with the spacetime metric $^{(4)}g_{\mu\nu}$. The spacetime metric transforms tensorially under this transformation, i.e., $^{(4)}g_{\mu' \nu'} =~^{(4)}g_{\mu\nu}\frac{\partial x^{\mu}}{\partial x^{\mu'}}\frac{\partial x^{\nu}}{\partial x^{\nu'}}$. Plug it into Eq.~(\ref{spacetime metric 3 + 1}), we have
\begin{equation}\label{spacetime metric}
^{\left(4\right)}g_{\mu'\nu'} = \left(g_{ab}X^{a}_{\mu}X^{b}_{\nu} - n_{\mu}n_{\nu}\right)\frac{\partial x^{\mu}}{\partial x^{\mu'}}\frac{\partial x^{\nu}}{\partial x^{\nu'}}
\end{equation}
By substituting indices $\mu'$ and $\nu'$ for $a'$ and $b'$ respectively in Eq.~(\ref{spacetime metric}), we can obtain the transformation rule of the spatial metric as following
\begin{align}\label{spatialmetric}
g_{a'b'} & = \left(g_{ab}X^{a}_{\mu}X^{b}_{\nu} - n_{\mu}n_{\nu}\right)\frac{\partial x^{\mu}}{\partial x^{a'}}\frac{\partial x^{\nu}}{\partial x^{b'}}\notag\\
&= g_{ab}\left(\frac{\partial x^{a}}{\partial x^{a'}} + \beta^{a}\frac{\partial t}{\partial x^{a'}}\right)\left(\frac{\partial x^{b}}{\partial x^{b'}} + \beta^{b}\frac{\partial t}{\partial x^{b'}}\right) - \alpha^2\frac{\partial t}{\partial x^{a'}}\frac{\partial t}{\partial x^{b'}}\notag\\
& = g_{ab}\frac{\partial x^{a}}{\partial x^{a'}}\frac{\partial x^{b}}{\partial x^{b'}} \ .
\end{align}
According to the result above, the spatial metric $g_{ab}$ transforms as a type $0 \choose 2$ tensor under the spatial diffeomorphism and a scalar under the time reparameterization.

And its inverse transforms as
\begin{equation}\label{spatialmetricinverse}
g^{a'b'} = g^{ab}\frac{\partial x^{a'}}{\partial x^{a}}\frac{\partial x^{b'}}{\partial x^{b}} \ , 
\end{equation}
which is a type $2 \choose 0$ tensor under the spatial diffeomorphism and a scalar under the time reparameterization. 

Let $g$ denote the determinant of the spatial metric $g_{ab}$. It transforms as
\begin{equation}\label{spatialmetricdet}
g' = \left|g_{a'b'}\right| = \left|g_{ab}\frac{\partial x^{a}}{\partial x^{a'}}\frac{\partial x^{b}}{\partial x^{b'}}\right|  = g\left|\frac{\partial x}{\partial x'}\right|^{2} \ ,
\end{equation}
where $\left|\partial x/\partial x'\right|$ stands for the Jacobian matrix determinant of this coordinate transformation. This shows that $g$ transforms as a weight +2 density under the spatial diffeomorphism and a scalar under the time reparameterization. 

With transformation rules of the spatial metric and its related quantities listed in Eqs.~(\ref{spatialmetric} - \ref{spatialmetricdet}), we move on to investigate transformation rules of the lapse function and shift vector. 

Equations [\ref{normal covector}-\ref{spacetime metric 3 + 1}] reveals that
\begin{align}
^{\left(4\right)}g_{a0} & = \beta_{a}\label{lapse}\\
^{\left(4\right)}g_{00} & = -\alpha^2 + g_{ab}\beta^{a}\beta^{b}\label{shift} \ .
\end{align}
According to Eq.~(\ref{lapse}) and Eq.~(\ref{spacetime metric}), we have
\begin{align}
\beta_{a'} & = \left(g_{ab}X^{a}_{\mu}X^{b}_{\nu} - n_{\mu}n_{\nu}\right)\frac{\partial x^{\mu}}{\partial x^{a'}}\frac{\partial x^{\nu}}{\partial t'}\notag\\
& = g_{ab}X^{a}_{\mu}X^{b}_{\nu}\delta_{c}^{\mu}\frac{\partial x^{c}}{\partial x^{a'}}\frac{\partial x^{\nu}}{\partial t'}\notag\\
& = g_{ab}\left(\delta^{b}_{\nu} + \beta^{b}\delta^{0}_{\nu}\right)\frac{\partial x^{a}}{\partial x^{a'}}\frac{\partial x^{\nu}}{\partial t'}\notag\\
& = \beta_{a}\frac{\partial x^{a}}{\partial x^{a'}}\frac{\partial t}{\partial t'} + g_{ab}\frac{\partial x^{a}}{\partial x^{a'}}\frac{\partial x^{b}}{\partial t'}\label{shiftcovectortransform}
\end{align}
Use the results from Eq.~(\ref{spatialmetric}) and Eq.~(\ref{shiftcovectortransform}), we can easily derive the transformation rule of the shift vector
\begin{align}
\beta^{a'} & = \beta_{b'}g^{a'b'}\notag\\
& = \left(\beta_{b}\frac{\partial x^{b}}{\partial x^{b'}}\frac{\partial t}{\partial t'} + g_{cd}\frac{\partial x^{d}}{\partial x^{b'}}\frac{\partial x^{c}}{\partial t'}\right)g^{ab}\frac{\partial x^{a'}}{\partial x^{a}}\frac{\partial x^{b'}}{\partial x^{b}}\notag\\
& = \beta^{a}\frac{\partial x^{a'}}{\partial x^{a}}\frac{\partial t}{\partial t'} + \frac{\partial x^{a'}}{\partial x^{a}}\frac{\partial x^{a}}{\partial t'}\label{shiftvectortransform}
\end{align}
From the calculation above we can tell that either the shift vector $\beta^{a}$ or its covector $\beta_{a}$ transforms tensorially under this change of coordinates. This issue can be addressed by subtracting the background shift vector from physical shift vector. The background shift vector transforms similarly as physical shift vector, i.e., 
\begin{equation}\label{backgroundshiftvector}
{\bar \beta}^{a'} = {\bar \beta}^{a}\frac{\partial x^{a'}}{\partial x^{a}}\frac{\partial t}{\partial t'} + \frac{\partial x^{a'}}{\partial x^{a}}\frac{\partial x^{a}}{\partial t'} \ .
\end{equation}
Eq.~(\ref{backgroundshiftvector}) yields that the difference between physical and background shift vector $\Delta \beta^{a} \equiv \beta^{a} - {\bar \beta}^{a}$ transforms as a type $1 \choose 0$ tensor under spatial diffeomorphism and a weight 1 density under time reparameterization, 
\begin{equation}\label{covariantshiftvector}
	\Delta \beta^{a'} = \Delta \beta^{a}\frac{\partial x^{a'}}{\partial x^{a}}\frac{\partial t}{\partial t'} \ ,
\end{equation}
and it serves as the covariant version of $\beta^{a}$. 

Use the result from Eq.~(\ref{shiftvectortransform}) along with Eq.~(\ref{lapse}) and Eq.~(\ref{spacetime metric}), we can derive the transformation rule of the lapse function. 
\begin{align*}
\alpha'^{2} & = - \left(g_{ab}X^{a}_{\mu}X^{b}_{\nu} - n_{\mu}n_{\nu}\right)\frac{\partial x^{\mu}}{\partial t'}\frac{\partial x^{\nu}}{\partial t'} + g_{a'b'}\beta^{a'}\beta^{b'}\notag\\ 
& = \alpha^2\left(\frac{\partial t}{\partial t'}\right)^{2} \ .
\end{align*}
Therefore we have
\begin{equation}\label{lapsetransform}
\alpha' = \alpha\frac{\partial t}{\partial t'}
\end{equation}
and hence the lapse function transforms as a scalar under spatial diffeomorphism and a weight +1 density under time reparameterization. 

Now transformation rules of significant $3 + 1$ terms with zero-order time or spatial derivatives are written down above. However, since partial differential equations play an important role in all the numerical recipes for solving Einstein's equations, we still seek to understand how partial derivative terms transform under the foliation preserving coordinate transformation. 

In order to study how partial derivatives transform, let's review the chain rule for this transformation briefly. The chain rule in spacetime domain states as below
\begin{equation}
\partial_{\mu'} = \frac{\partial x^{\mu}}{\partial x^{\mu'}}\partial_{\mu} \ .
\end{equation}
Therefore, if $\mu' = t'$, we have that
\begin{equation}\label{timederivative}
\partial_{t'} = \frac{\partial x^{c}}{\partial t'}\partial_{c} + \frac{\partial t}{\partial t'}\partial_{t}
\end{equation}
and if $\mu' = a'$, with $\partial t/\partial x^{a'} = 0$, we have
\begin{equation}\label{spatial derivative}
\partial_{a'} = \frac{\partial x^{c}}{\partial x^{a'}}\partial_{c} \ .
\end{equation}
From Eq.~(\ref{timederivative}) and Eq.~(\ref{spatial derivative}) we can tell that the spatial derivative transforms as a type $0 \choose 1$ tensor without introducing any extra terms. On the other hand, in the time derivative transformations, there are one extra term $\left(\partial t/\partial t'\right)\partial_{t}$ involved, since $\partial t/\partial t' \ne 0$. According to this analysis, the spatial Christoffel symbols transform as its spacetime counterpart under spacetime coordinate transformation, due to the fact that it is only constructed by the spatial metric and its spatial derivatives, 
\begin{equation}
\Gamma^{a'}_{~b'c'} = \Gamma^{a}_{~bc}\frac{\partial x^{a'}}{\partial x^{a}}\frac{\partial x^{b}}{\partial x^{b'}}\frac{\partial x^{c}}{\partial x^{c'}} + \frac{\partial x^{a'}}{\partial x^{d}}\frac{\partial^{2} x^{d}}{\partial x^{b'}\partial x^{c'}} \ .
\end{equation}
The covariant spatial Christoffel symbols can be easily obtained as $\Delta \Gamma^{a}_{~bc} \equiv \Gamma^{a}_{~bc} - {\bar \Gamma}^{a}_{~bc}$ and it transforms as a type $1 \choose 2$ tensor under spatial diffeomorphism and a scalar under time reparameterization. 
\begin{equation}
	\Delta \Gamma^{a'}_{~b'c'} = \Delta \Gamma^{a}_{~bc}\frac{\partial x^{a'}}{\partial x^{a}}\frac{\partial x^{b}}{\partial x^{b'}}\frac{\partial x^{c}}{\partial x^{c'}} \ .	
\end{equation}

Transformation of shift covector's spatial covariant derivative can be derived as following, 
\begin{align}\label{shiftvectorcovariantspatialderivativetransform}
D_{a'}\beta_{b'} &= D_{a'}\left(\beta^{b}\frac{\partial x^{b}}{\partial x^{b'}}\frac{\partial t}{\partial t'}\right) + D_{a'}\left(g_{bc}\frac{\partial x^{b}}{\partial x^{b'}}\frac{\partial x^{c}}{\partial t'}\right)\notag\\
& = \frac{\partial t}{\partial t'}D_{a'}\left(\beta_{b}\frac{\partial x^{b}}{\partial x^{b'}}\right) + D_{a'}\left(g_{b'c'}\frac{\partial x^{c'}}{\partial x^{c}}\frac{\partial x^{c}}{\partial t'}\right)\notag\\
& = D_{a}\beta^{b}\frac{\partial x^{a}}{\partial x^{a'}}\frac{\partial x^{b}}{\partial x^{b'}}\frac{\partial t}{\partial t'} + g_{b'c'}D_{a'}\left(\frac{\partial x^{c'}}{\partial x^{c}}\frac{\partial x^{c}}{\partial t'}\right) \ .
\end{align} 
It is obvious that general covariance of this transformation can be achieved by replacing $\beta_{a}$ by its covariant counterpart $\Delta \beta_{a}$. It is a type $0 \choose 2$ tensor under spatial diffeomorphism and a weight 1 scalar density under time reparameterization, 
\begin{equation}\label{covariantshiftvectorcovariantspatialderivativetransform}
	D_{a'}\Delta\beta_{b'} = D_{a}\Delta\beta_{b}\frac{\partial x^{a}}{\partial x^{a'}}\frac{\partial x^{b}}{\partial x^{b'}}\frac{\partial t}{\partial t'} \ .
\end{equation}	
The spatial covariant derivative of $\alpha$, which takes form of a regular spatial partial derivative, transforms as 
\begin{align}\label{lapsefunctionspatialderivative}
\partial_{a'}\alpha' & = \partial_{a'}\left(\alpha \frac{\partial t}{\partial t'}\right) \notag \\
& = \frac{\partial t}{\partial t'}\partial_{a'}\alpha \notag \\
& = \frac{\partial t}{\partial t'}\frac{\partial x^{c}}{\partial x^{a'}}\partial_{c}\alpha \ .
\end{align}
It is a type $0 \choose 1$ tensor under spatial diffeomorphism and a weight +1 density under time reparameterization.

Due to the extra terms introduced by Eq.~(\ref{timederivative}), transformation of time derivatives of $\alpha$ posts a more interesting case. Let's first examine the time derivative of spatial metric. It transforms as following, 
\begin{align}
\partial_{t'}g_{a'b'} & = \left(\partial_{t'}g_{ab}\right)\frac{\partial x^{a}}{\partial x^{a'}}\frac{\partial x^{b}}{\partial x^{b'}} + g_{ab}\partial_{t'}\left(\frac{\partial x^{a}}{\partial x^{a'}}\frac{\partial x^{b}}{\partial x^{b'}}\right)\notag\\
& =  \left(\partial_{t}g_{ab}\right)\frac{\partial t}{\partial t'}\frac{\partial x^{a}}{\partial x^{a'}}\frac{\partial x^{b}}{\partial x^{b'}} + \left(\partial_{c}g_{ab}\right)\frac{\partial x^{c}}{\partial t'}\frac{\partial x^{a}}{\partial x^{a'}}\frac{\partial x^{b}}{\partial x^{b'}} + g_{ab}\partial_{t'}\left(\frac{\partial x^{a}}{\partial x^{a'}}\frac{\partial x^{b}}{\partial x^{b'}}\right) \ .
\end{align}
Hereby we introduce the notion of {\em covariant time derivative} under foliation preserving coordinate transformation, $D_{t}$, such that $D_{t}T^{A}$ transforms tensorially, where $T^{A}$ is an arbitrary tensor density whose indices are contracted as $A$. For instance, we require covariant time derivative of the spatial metric transforms as
\begin{equation}
	D_{t'}g_{a'b'} = D_{t}g_{ab}\frac{\partial x^{a}}{\partial x^{a'}}\frac{\partial x^{b}}{\partial x^{b'}}\frac{\partial t}{\partial t'} \ .
\end{equation}
By revisiting the definition of extrinsic curvature in Eq.~(\ref{extrinsic}), a keen observer would notice that $D_{t}g_{ab}$ should be closely related to the extrinsic curvature $K_{ab}$. Since $K_{ab}$ is a pure spatial tensor, it transforms tensorially
\begin{equation}
	K_{a'b'} = K_{ab}\frac{\partial x^{a}}{\partial x^{a'}}\frac{\partial x^{b}}{\partial x^{b'}} \ .
\end{equation}
Hence we can define the covariant time derivative of spatial metric $D_{t}g_{ab}$ as
\begin{equation}\label{spatialmetriccovarianttimederivative}
	D_{t}g_{ab} \equiv (\partial_{t} - \mathcal{L}_{\beta})g_{ab} \ .
\end{equation}
Furthermore, it is straightforward to verify that for any quantity that transforms as a scalar under time reparameterization, the covariant time derivative operator can be defined as
\begin{equation}
	D_{t} \equiv \partial_{t} - \mathcal{L}_{\beta} \ .
\end{equation}
Covariant time derivative for lapse function and shift vector is more complicated, since they don't reside entirely in the manifold of a single $3 + 1$ foliation. Therefore we are required to study the covariant time derivative of these two quantities more carefully. 

We start by writing down transformation rule of regular time derivative of the lapse function, as following
\begin{align}\label{lapsetimederivative}
\partial_{t'}\alpha' & = \partial_{t'}\left(\alpha\frac{\partial t}{\partial t'}\right)\notag\\
& = \frac{\partial t}{\partial t'}\partial_{t'}\alpha + \alpha\frac{\partial^{2}t}{\partial {t'}^{2}}\notag\\
& = \partial_{t}\alpha\left(\frac{\partial t}{\partial t'}\right)^{2} + \partial_{c}\alpha\frac{\partial x^{c}}{\partial t'}\frac{\partial t}{\partial t'} + \alpha\frac{\partial^{2}t}{\partial {t'}^{2}} \ .
\end{align}
The leading term in Eq.~(\ref{lapsetimederivative}) reveals that covariant time derivative of $\alpha$ should transform as a scalar under spatial diffeomorphism and a weight +2 density under time reparamterization. According to transformation rules listed in Eqs.~(\ref{shiftvectortransform}, \ref{lapsetransform}, \ref{lapsefunctionspatialderivative}), we are able to construct the covariant time derivative of lapse function as
\begin{equation}\label{lapsecovarianttimederivative}
	D_{t}\alpha \equiv \partial_{t}\alpha - \beta^{c}\partial_{c}\alpha - \frac{\alpha}{{\bar \alpha}}\left(\partial_{t}{\bar \alpha} - {\bar \beta}^{c}\partial_{c}{\bar \alpha}\right) 
\end{equation}
and one can check that its transformation rule turns out to be exactly as expected
\begin{equation}
	D_{t'}\alpha' = D_{t}\alpha \left(\frac{\partial t}{\partial t'}\right)^{2} \ .
\end{equation}

Since $\Delta \beta^{a}$ is a covariant version of $\beta^{a}$, the covariant time derivative of shift vector is easier to be constructed from $\partial_{t}\Delta \beta^{a}$. The transformation rule of $\partial_{t}\Delta \beta^{a}$ can be written as following
\begin{align}\label{covariantshifttimederivative} 
\partial_{t'}\Delta \beta^{a'} & = \left(\partial_{t'}\Delta \beta^{a}\right)\frac{\partial x^{a'}}{\partial x^{a}}\frac{\partial t}{\partial t'} + \Delta \beta^{a}\partial_{t'}\left(\frac{\partial x^{a'}}{\partial x^{a}}\frac{\partial t}{\partial t'}\right)\notag\\
& = \partial_{t}\Delta \beta^{a}\frac{\partial x^{a'}}{\partial x^{a}}\left(\frac{\partial t}{\partial t'}\right)^{2} + \partial_{c}\Delta \beta^{a}\frac{\partial x^{c}}{\partial t'}\frac{\partial x^{a'}}{\partial x^{a}}\frac{\partial t}{\partial t'}\notag\\
& + \Delta \beta^{a}\frac{\partial x^{a'}}{\partial x^{a}}\frac{\partial^{2}t}{\partial t'^{2}} + \Delta \beta^{a}\frac{\partial^{2} x^{a'}}{\partial x^{c}\partial x^{a}}\frac{\partial x^{c}}{\partial t'}\frac{\partial t}{\partial t'} + \Delta \beta^{a}\frac{\partial^{2}x^{a'}}{\partial t\partial x^{a}}\left(\frac{\partial t}{\partial t'}\right)^{2} \ .
\end{align}
Based on transformation rules of shift vector and its covariant spatial derivative, as listed in Eqs.~(\ref{shiftvectortransform}, \ref{shiftvectorcovariantspatialderivativetransform}, \ref{covariantshiftvectorcovariantspatialderivativetransform}), we propose an ansatz of $D_{t}\beta^{a}$ as following
\begin{equation}
	D_{t}\beta^{a} \equiv \partial_{t}\Delta \beta^{a} - \frac{\Delta \beta^{a}}{{\bar \alpha}}\left(\partial_{t}{\bar \alpha} - {\bar \beta}^{a}\partial_{a}{\bar \alpha}\right) + c_{1}\beta^{b}{\bar D}_{b}\beta^{a} + c_{2}\beta^{b}{\bar D}_{b}{\bar \beta}^{a} + c_{3}{\bar \beta}^{b}{\bar D}_{b}\beta^{a} + c_{4}{\bar \beta}^{b}{\bar D}_{b}{\bar \beta}^{a}	\ ,
\end{equation}
where $c_{1}$, $c_{2}$, $c_{3}$, $c_{4}$ are constants to be determined; meanwhile we require it to transform as
\begin{equation}
	D_{t'}\beta^{a'} = D_{t}\beta^{a}\frac{\partial x^{a'}}{\partial x^{a}}\left(\frac{\partial t}{\partial t'}\right)^{2} \ ,
\end{equation}
a type $1 \choose 0$ tensor under spatial diffeomorphism and a weight +2 density under time reparameterization. 

The process of determining values of these four constants is tedious and will bore the readers if they are not bored already, so we skip directly to the conclusion here. In order to achieve the covariant time derivative of shift vector, the constraints $c_{1}$, $c_{2}$, $c_{3}$ and $c_{4}$ have to satisfy are
\begin{align*}
c_{1} + c_{3} & = -1\\
c_{2} + c_{4} & = 1\\
c_{1} + c_{2} & = 1\\
c_{3} + c_{4} & = -1 \ .
\end{align*}
Define $\sigma \equiv c_{1}$, then we can write all the other constants in terms of $\sigma$ and represent $D_{t}\beta^{a}$ as
\begin{equation}
	D_{t}\beta^{a} \equiv \partial_{t}\Delta \beta^{a} - \frac{\Delta \beta^{a}}{{\bar \alpha}}\left(\partial_{t}{\bar \alpha} - {\bar \beta}^{a}\partial_{a}{\bar \alpha}\right) + \sigma\beta^{b}{\bar D}_{b}\beta^{a} + \left(1-\sigma\right)\beta^{b}{\bar D}_{b}{\bar \beta}^{a} - \left(1 + \sigma\right){\bar \beta}^{b}{\bar D}_{b}\beta^{a} + \sigma{\bar \beta}^{b}{\bar D}_{b}{\bar \beta}^{a} \ .
\end{equation}
In both definitions of $D_{t}g_{ab}$ and $D_{t}\alpha$, the covariant time derivative operators share the same pattern as $\partial_{t} - \beta^{b}\partial_{b}$. In order to achieve the same pattern in $D_{t}\beta^{a}$, an appropriate choice for $\sigma$ is $\sigma = -1$. Therefore, in the rest of this paper, the covariant time derivative of shift vector is defined as
\begin{equation}\label{shiftcovarianttimederivative}
	D_{t}\beta^{a} \equiv \partial_{t}\Delta \beta^{a} - \frac{\Delta \beta^{a}}{{\bar \alpha}}\left(\partial_{t}{\bar \alpha} - {\bar \beta}^{b}\partial_{b}{\bar \alpha}\right) - \beta^{b}{\bar D}_{b}\Delta \beta^{a} + \Delta \beta^{b}{\bar D}_{b}{\bar \beta}^{a} \ .	
\end{equation}

Transformation rule for the gauge source term of generalized harmonic formalism needs to be examined as well, but first we need to define its 3 + 1 decomposition. The decomposition splits the spacetime gauge source term into a spatial component
\begin{equation}\label{spatialgaugesource}
	H_{a} \equiv H_{\mu}X^{\mu}_{a} \ ,
\end{equation}
and a timelike component
\begin{equation}
	H_{\perp} \equiv H_{\mu}n^{\mu} \ .
\end{equation}
According to the definition in Eq.~(\ref{spatialgaugesource}), $H_{a}$ transforms as a $0 \choose 1$ tensor under spatial diffeomorphism and a scalar under time reparameterization, 
\begin{align}
H_{a'} & = H_{\mu'}X^{\mu'}_{a'}\\
& = H_{\mu}\frac{\partial x^{\mu}}{\partial x^{\mu'}}\delta^{\mu'}_{a'}\\
& = H_{a}\frac{\partial x^{a}}{\partial x^{a'}} \ .
\end{align}
In addition, we can easily conclude that $H^{a}$ transforms 
\begin{equation}
H^{a'} = H^{a}\frac{\partial x^{a'}}{\partial x^{a}} \ ,
\end{equation}
a $1 \choose 0$ tensor under spatial diffeomorphism and a scalar under time reparameterization. 

For $H_{\perp}$, we have
\begin{align}
H_{\perp'} & = H_{\mu'}n^{\mu'} \notag\\
& = H_{\mu'}\left(\delta^{\mu'}_{0'} - \beta^{c'}\delta_{c'}^{\mu'}\right)/\alpha'\notag\\
& = H_{\mu}\frac{\partial x^{\mu}}{\partial x^{\mu'}}\left(\delta^{\mu'}_{0'} - \beta^{c}\frac{\partial x^{c'}}{\partial x^{c}}\frac{\partial t}{\partial t'}\delta^{\mu'}_{c'} - \frac{\partial x^{c'}}{\partial x^{c}}\frac{\partial x^{c}}{\partial t'}\delta^{\mu'}_{c'}\right)\frac{\partial t'}{\partial t} / \alpha \notag\\
& = H_{\mu}\left(\delta^{\mu}_{0} - \beta^{c}\delta_{c}^{\mu}\right)/\alpha \notag\\
& = H_{\mu}n^{\mu} \notag\\
& = H_{\perp} \ .
\end{align}
Hence we have that $H_{\perp}$ acts as a scalar under both spatial diffeomorphism and time reparemeterization. 

As it is discussed before, any quantity transforms as a scalar under time reparameterization, its covariant time derivative operator can be expressed as $D_{t} \equiv \partial_{t} - \mathcal{L}_{\beta}$. Therefore, for covariant time derivative of $H_{\perp}$ and $H_{a}$, we have that
\begin{subequations}
	\begin{align}
		D_{t}H_{\perp} &\equiv \left(\partial_{t} - \mathcal{L}_{\beta}\right)H_{\perp}\\
		D_{t}H_{a} &\equiv \left(\partial_{t} - \mathcal{L}_{\beta}\right)H_{a} \ .
	\end{align}
\end{subequations}

Hereby we conclude the study of transformation rules under foliation preserving coordinate transformation. 

%%%%%%%%%%%%%%%%%%%%%%%%%%%%%%%
\section{Covariant Gauge Condition}\label{gauge}
With transformation rules of all important physical variables under foliation preserving coordinate transformation listed in Sec.~\ref{transform}, we are now able to revisit the non-covariant gauge condition problems mentioned in Sec.~\ref{generalcovariancefoliationpreservingtransformation}. In this section, we try to recover general covariance of various gauge conditions used in different numerical relativity formulations. The strategy for making gauge conditions covariant has two steps. First, replace non-covariant terms with their covariant counterparts. For instance, replace $\partial_{t}\alpha$ with $D_{t}\alpha$ or substitute $\partial_{t}\beta^{a}$ with $D_{t}\beta^{a}$. The second step is balancing tensor density weights on both sides of the equation. Multiplying by ${\bar \alpha}$ increases weight density by 1 under time reparameterization while multiplying by $\sqrt{{\bar g}}$ increase weight density by 1 under spatial diffeomorphism. This strategy is illustrated in Sec.~\ref{movingpuncture} and Sec.~\ref{dampedwave} while Sec.~\ref{movingpuncture} discusses the gauge condition used in BSSN formalism and Sec.~\ref{dampedwave} presents a general covariant representation of the damped-wave gauge condition\cite{Pretorius:2006tp, Szilagyi:2009qz, Lindblom:2009tu} employed by the generalized harmonic community. 
\subsection{Moving puncture gauge condition}\label{movingpuncture}
Moving puncture gauge condition consists of two parts, the 1 + log slicing condition governs evolution of the lapse function and Gamma-driver shift condition for the shift vector. 

1 + log slicing condition reads as
\begin{equation}\label{1 + log slicing}
\partial_{t}\alpha - \beta^{c}\partial_{c}\alpha = -2\alpha K \ .
\end{equation}
As we can notice that the left hand side shares several terms with $D_{t}\alpha$, it is intuitive and appropriate to replace the left hand side with $D_{t}\alpha$. However, this substitution make the left hand side transforms as a weight +2 density under time reparameterization while the right hand side transforms as a weight +1 density under such transformation. Therefore, we multiply by the background lapse function ${\bar \alpha}$, which transforms as a weight +1 density, on the right hand side to balance the equation's density weights. This leads to the general covariant 1 + log slicing condition, as following
\begin{equation}\label{covariant 1 + log slicing}
D_{t}\alpha = -2\alpha{\bar \alpha}K \ .
\end{equation}

The Gamma-driver shift condition is usually written as a system of two first order equations
\begin{subequations}\label{gammadrivershiftset}
\begin{align}
\partial_{t}\beta^{a} & = \beta^{b}\partial_{b}\beta^{a} + \frac{3}{4}B^{a}\label{gammadrivershiftfirst}\\
\partial_{t}B^{a} & = \beta^{b}\partial_{b}B^{a} + \partial_{t}{\tilde \Gamma}^{a} - \beta^{b}\partial_{b}{\tilde \Gamma}^{a} - \eta B^{a}\label{gammadrivershiftsecond}
\end{align}
\end{subequations}
where ${\tilde \Gamma}^{a}$ is the trace of the conformal Christoffel symbols defined as
\begin{equation}
{\tilde \Gamma}^{a} \equiv \sqrt{g}^{2/3}\left(\Gamma^{a}_{~bc}g^{bc} + \frac{1}{3}g^{ab}\Gamma^{c}_{~bc}\right)
\end{equation}
and $\eta$ is a constant parameter. 

We can follow the same strategy to adjust the Gamma-driver shift condition, replacing $\partial_{t}\beta^{a}$ by $D_{t}\beta^{a}$, changing every non-tensorial term to its corresponding tensorial term and balance the density weight on both sides. 

Manipulating Eq.~(\ref{gammadrivershiftfirst}) in the following way could recover its general covariance while restricting $B^{a}$ to transform as a type $1 \choose 0$ tensor under spatial diffeomorphism and a scalar under time reparameterization, 
\begin{equation}\label{gammadrivershiftfirstcovariant}
	D_{t}\beta^{a} = \frac{3}{4}{\bar \alpha}^{2}B^{a} \ .
\end{equation}

For Eq.~(\ref{gammadrivershiftsecond}), we first need to construct a covariant form of ${\tilde \Gamma}^{a}$, which is defined as
\begin{equation}
\Delta {\tilde \Gamma}^{a} \equiv \sqrt{\frac{g}{{\bar g}}}^{2/3}\left(\Delta \Gamma^{a}_{~bc}g^{bc} + \frac{1}{3}g^{ab}\Delta \Gamma^{c}_{~bc}\right) \ ,
\end{equation}
so that it transforms exactly the same as $B^{a}$. Then the covariant form of Eq.~(\ref{gammadrivershiftsecond}) can be written as
\begin{equation}\label{gammadrivershiftsecondcovariant}
	D_{t}B^{a} - D_{t}\Delta {\tilde \Gamma}^{a}  = - \eta {\bar \alpha}B^{a} \ , 
\end{equation}
where
\begin{subequations}
\begin{align}
D_{t}B^{a} & \equiv \left(\partial_{t} - \mathcal{L}_{\beta}\right)B^{a}\\
D_{t}{\tilde \Gamma}^{a} & \equiv \left(\partial_{t} - \mathcal{L}_{\beta}\right){\tilde \Gamma}^{a} \ .
\end{align}
\end{subequations}
In summary, Eqs.~(\ref{gammadrivershiftfirstcovariant}, \ref{gammadrivershiftsecondcovariant}) represent general covariant Gamma-driver shift condition as a system of two first order equations. 

On the other side, in some literatures \cite{vanMeter:2006vi}, Eqs.~(\ref{gammadrivershiftset}) can be integrated with initial conditions $\beta^{a}\left(t=0\right) = B^{a}\left(t=0\right) = {\tilde \Gamma}^{a}\left(t = 0\right) = 0$ and yields a single-equation form of Gamma-driver shift condition
\begin{equation}\label{gammadrivershift}
\partial_{t}\beta^{a} = \beta^{b}\partial_{b}\beta^{a} + \frac{3}{4}{\tilde \Gamma}^{a} - \eta \beta^{a} \ .
\end{equation} 
Similarly, we can write the covariant form of Eq.~(\ref{gammadrivershift}) as
\begin{equation}\label{covariant Gamma-driver shift}
D_{t}\beta^{a} = \frac{3}{4}{\bar \alpha}^{2}\Delta {\tilde \Gamma}^{a} - \eta {\bar \alpha}\Delta\beta^{a} \ .
\end{equation}
%Due to its simplicity, we focus our discussion on Eq.~(\ref{covariant Gamma-driver shift}) instead of Eqs.~(\ref{covariant Gamma-driver shift set}) in the rest of this paper.

%Writing down these covariant gauge conditions creates more possibilities for numerical simulation. When the need of evolving the system in a different coordinates system rises, one can just use the following equations as gauge conditions
%\begin{align*}
%D_{t'}\alpha' & = -2\alpha'{\bar \alpha}'K'\\
%D_{t'}\beta^{a'} & = \frac{3}{4}{\bar \alpha}'\Delta {\tilde \Gamma}^{a'} - \eta {\bar \alpha}'\Delta\beta^{a'}
%\end{align*}
%while the transformed background metric terms are associated with the original terms as below
%\begin{align}
%{\bar g}_{a'b'} & = {\bar g}_{ab}\frac{\partial x^{a}}{\partial x^{a'}}\frac{\partial x^{b}}{\partial x^{b'}}\\
%{\bar g'} & = {\bar g}\left|\frac{\partial x}{\partial x'}\right|^{2}\\
%{\bar \alpha}' & = {\bar \alpha}\frac{\partial t}{\partial t'}\\
%{\bar \beta}^{a'} & = {\bar \beta}^{a}\frac{\partial x^{a'}}{\partial x^{a}}\frac{\partial t}{\partial t'} + \frac{\partial x^{a'}}{\partial x^{c}}%\frac{\partial x^{c}}{\partial t'}\label{background shift}\\
%{\bar \Gamma}^{a'}_{~b'c'} & = {\bar \Gamma}^{a}_{~bc}\frac{\partial x^{a'}}{\partial x^{a}}\frac{\partial x^{b}}{\partial x^{b'}}\frac{\partial x^{c}}{\partial x^{c'}} + \frac{\partial x^{a'}}{\partial x^{d}}\frac{\partial x^{d}}{\partial x^{b'}\partial x^{c'}}
%\end{align}

\subsection{Damped-wave gauge condition}\label{dampedwave}
By contrast, due to the fact that the gauge source terms $H^{a}$ and $H_{\perp}$ already transform tensorially, the damped-wave gauge condition is much easier to adjust. What we need to do is merely balancing density weights on both sides of the equations. 

Let us begin by considering the damped-wave gauge condition for $H^{a}$
\begin{equation}\label{damped gauge vector}
H^{a} = -\mu_{S}\frac{\beta^{a}}{\alpha}
\end{equation}
where $\mu_{S}$ is a numerical damping factor. 

The only non-tensor term in Eq.~(\ref{damped gauge vector}) is the shift vector $\beta^{a}$ and we can easily fix it by replacing $\beta^{a}$ with $\Delta \beta^{a}$. Since both $\Delta \beta^{a}$ and $\alpha$ act as a weight +1 tensor density under time reparameterization, $\Delta \beta^{a}/\alpha$ simply transforms as a scalar under time reparameterization, which is the same case for $H^{a}$. Therefore, we can write the covariant form of Eq.~(\ref{damped gauge vector}) as
\begin{equation}\label{covariant damped gauge vector}
H^{a} = -\mu_{S}\frac{\Delta \beta^{a}}{\alpha} \ .
\end{equation}
For the other gauge source term $H_{\perp}$, the traditional gauge condition is
\begin{equation}\label{damped gauge scalar}
H_{\perp} = -\mu_{L}\log\left(\frac{\sqrt{g}}{\alpha}\right) \ .
\end{equation}
Since $H_{\perp}$ transforms as a scalar under both time-dependent spatial diffeomorphism and time reparameterization, it is obvious that we can modify Eq.~(\ref{damped gauge scalar}) as following
\begin{equation}\label{covariant damped gauge scalar}
H_{\perp} = -\mu_{L}\log\left(\frac{\sqrt{g/{\bar g}}}{\alpha/{\bar \alpha}}\right)
\end{equation} 
so that it becomes covariant. 
%%%%%%%%%%%%%%% I am marching forward %%%%%%%%%%%%%%%%%%
\subsection{Another way to achieve covariant gauge condition}
We can also achieve the transformed gauge conditions from the regular gauge conditions. All we need to do is using the transformation rules we derived above to write the non-transformed variables in terms of the transformed terms. And we argue as following that these two methods should yield identical transformed gauge conditions. 

Assume we have some variables $v$. Under a change of coordinates, they transform to $v' = T\left(v\right)$ where $T$ is some function. We also introduce a 
set of background fields that transform in the same way; thus $\bar v' = T\left(\bar v\right)$. The gauge conditions in covariant form are $F^A\left(v,\bar v\right) = 0$, 
where the index $A$ ranges over the number of conditions. Since the gauge conditions are covariant, we have 
\begin{equation}
	F^A\left(v',\bar v'\right) = \xi^A_B F^B\left(v,\bar v\right)
\end{equation}
where $\xi^A_B$ is an invertible matrix. That is, 
\begin{equation}\label{covariant equivalence}
	 F^A\left(v,\bar v\right) = 0 \Longleftrightarrow F^A\left(v',\bar v'\right) = 0
\end{equation}
In the original coordinate system, the gauge condition is $F^A\left(v,\bar v\right) = 0$ and in the transformed coordinate system, the 
gauge condition is $F^A\left(v',\bar v'\right) = 0$.

Now we want to consider a particular background. Assume in the original coordinate system, the background fields take the values $\bar v = \bar c$. The gauge 
conditions in original coordinates are $F^A\left(v,\bar c\right) = 0$. (This is the non--covariant version of the gauge conditions.) As we discussed above, we have two ways to achieve the transformed gauge conditions. One way: 
\begin{itemize}
	\item Start with the gauge written in covariant form, $F^A\left(v,\bar v\right) = 0$.
	\item Change to transformed coordinates, where the gauge is $F^A\left(v',\bar v'\right) = 0$.
	\item Set the background to $\bar v = \bar c$. But $\bar v' = T\left(\bar v\right)$ so that $\bar v' = T\left(\bar c\right)$. 
	\item Then the gauge conditions in transformed coordinates read $F^A\left(v',T\left(\bar c\right)\right) = 0$. 
\end{itemize}
Another way: 
\begin{itemize}
	\item Start with the gauge written in covariant form, $F^A\left(v,\bar v\right) = 0$.
	\item Set the background to $\bar v = \bar c$, so the gauge takes the non--covariant form $F^A\left(v,\bar c\right) = 0$. 
	\item Now use the transformation rules to replace the original fields with transformed fields; that is, use $v = T^{-1}\left(v'\right)$. 
	The gauge becomes $F^A\left(T^{-1}\left(v'\right),\bar c\right) = 0$. 
\end{itemize}
These two results are equivalent. That is, 
\begin{equation}
	F^A\left(T^{-1}\left(v'\right),\bar c\right) = 0 \Longleftrightarrow F^A\left(v',T\left(\bar c\right)\right) = 0
\end{equation}
This is guaranteed by the Eq.~(\ref{covariant equivalence}) above. In Eq.~(\ref{covariant equivalence}), just let $v = T^{-1}\left(v'\right)$ and $\bar v = \bar c$, which implies $\bar v' = T\left(\bar c\right)$.

We will explain how this scheme works in detail with a specific example in the next section. 

%%%%%%%%%%%%%%%%%%%%%%%%%%%%%
\section{Time Dependent Spatial Rotation Coordinate Transformation}\label{example}
In this section, we examine a specific kind of coordinate transformation, the time dependent spatial rotation coordinate transformation, and demonstrate how the covariant gauge conditions work in action. 

In this coordinate transformation, we keep the $t$ coordinate intact while let the spatial coordinate rotates around the $z$ axis at a constant angular speed $\omega$. Hence the transformed coordinates can be expressed in terms of the original terms as below
\begin{align}
t' & = t\notag\\
x' & = x \cos\left(\omega t\right) + y\sin\left(\omega t\right)\notag\\
y' & = -x\sin\left(\omega t\right) + y\cos\left(\omega t\right)\notag\\
z' & = z
\end{align}
and equivalently, 
\begin{align}
t & = t'\notag\\
x & = x'\cos\left(\omega t'\right) - y'\sin\left(\omega t'\right)\notag\\
y & = x'\sin\left(\omega t'\right) + y'\cos\left(\omega t'\right)\notag\\
z & = z'
\end{align}
We also assume that the background metric is trivial
\begin{align}
{\bar g}_{ab} & = diag\left(1,1,1\right)\notag\\
{\bar \alpha} & = 1\notag\\
{\bar \beta} & = 0 
\end{align}
Under this transformation, let us first review the moving puncture gauge condition. Since $\partial t/\partial t' = 1$, ${\bar \alpha}' = {\bar \alpha} = 1$ and hence the covariant time derivative of $\alpha$ will collapse to
\begin{equation}
D_{t}\alpha = \partial_{t}\alpha - \beta^{a}\partial_{a}\alpha
\end{equation}
Then the covariant 1 + log slicing condition under this transformation becomes
\begin{equation}\label{transformed 1 + log slicing}
\partial_{t'}\alpha' - \beta^{a'}\partial_{a'}\alpha' = -2\alpha'K'
\end{equation}
One would argue that this equation is identical to the regular 1 + log slicing equation. This is only true under this specific coordinate transformation, especially when we have $\partial t/\partial t' = 1$. One will see the difference in the Gamma-driver shift condition discussed below. 

According to Eq.~(\ref{background shift}) and the choice that ${\bar \beta}^{a} = 0$, we have that
\[
{\bar \beta}^{a'} = \frac{\partial x^{a'}}{\partial x^{a}}\frac{\partial x^{a}}{\partial t'}
\]
Therefore, for each component, we have
\begin{align*}
{\bar \beta}^{1'} & = -\omega y'\\
{\bar \beta}^{2'} & = \omega x'\\
{\bar \beta}^{3'} & = 0
\end{align*}
Similarly, for the background Christoffel symbols, we have
\[
{\bar \Gamma}^{a'}_{b'c'} = \frac{\partial x^{a'}}{\partial x^{d}}\frac{\partial x^{d}}{\partial x^{b'}\partial x^{c'}}
\]
however, since in this coordinate transformation, $x^{a}$ is a first order function of $x^{a'}$, ${\bar \Gamma}^{a'}_{b'c'}$ vanishes. 

Also, it is easy to check that $\left|\partial x/\partial x'\right| = 1$, this result yields that ${\bar g}' = {\bar g} = 1$. According to these results above, the transformed Gamma-driver shift condition becomes
\begin{equation}
\partial_{t'}\beta^{a'} - \beta^{b'}\partial_{b'}\beta^{a'} + 2\beta^{b'}\partial_{b'}{\bar \beta}^{a'} - {\bar \beta}^{b'}\partial_{b'}{\bar \beta}^{a'} = \frac{3}{4}{\tilde \Gamma}^{a'} - \eta \Delta \beta^{a'}
\end{equation}
and for each component, we have
\begin{align}
\partial_{t'}\beta^{1'} - \beta^{b'}\partial_{b'}\beta^{1'} - 2\omega\beta^{2'} + \omega^{2}x' &= \frac{3}{4}{\tilde \Gamma}^{1'} - \eta \left(\beta^{1'} + \omega y'\right)\label{transformed Gamma-driver shift x}\\
\partial_{t'}\beta^{2'} - \beta^{b'}\partial_{b'}\beta^{2'} + 2\omega\beta^{1'} + \omega^{2}y' &= \frac{3}{4}{\tilde \Gamma}^{2'} - \eta \left(\beta^{2'} - \omega x'\right)\label{transformed Gamma-driver shift y}\\
\partial_{t'}\beta^{3'} - \beta^{b'}\partial_{b'}\beta^{3'} & = \frac{3}{4}{\tilde \Gamma}^{3'} - \eta \beta^{3'}\label{transformed Gamma-driver shift z}
\end{align}
To verify the transformed gauge conditions(Eq.~(\ref{transformed 1 + log slicing}) and Eqs.~(\ref{transformed Gamma-driver shift x}-\ref{transformed Gamma-driver shift z})) obtained by the covariant gauge conditions, we try to write down the transformed gauge conditions by transforming the variables in the regular gauge conditions explicitly and compare with the equations we already have. 

According to the transformation rules we obtained in Sec.~\ref{transform}, let's write the untransformed variables in terms of the transformed variables. 
\begin{align}
\alpha & = \alpha'\frac{\partial t'}{\partial t} = \alpha'\label{alpha to alpha prime}\\
\beta^{a} & = \left(\beta^{a'} - \frac{\partial x^{a'}}{\partial x^{b}}\frac{\partial x^{b}}{\partial t'}\right)\frac{\partial x^{a}}{\partial x^{a'}}\frac{\partial t'}{\partial t} =   \left(\beta^{a'} - \frac{\partial x^{a'}}{\partial x^{b}}\frac{\partial x^{b}}{\partial t'}\right)\frac{\partial x^{a}}{\partial x^{a'}}\label{beta to beta prime}\\
g_{ab} & = g_{a'b'}\frac{\partial x^{a'}}{\partial x^{a}}\frac{\partial x^{b'}}{\partial x^{b}}\label{metric to metric prime}\\
g & = g'\left(\frac{\partial x'}{\partial x}\right)^{2} = g'\label{det to det prime}\\
\Gamma^{a}_{~bc} & = \left(\Gamma^{a'}_{~b'c'} - \frac{\partial x^{a'}}{\partial x^{d}}\frac{\partial x^{d}}{\partial x^{b'}\partial x^{c'}}\right)\frac{\partial x^{a}}{\partial x^{a'}}\frac{\partial x^{b'}}{\partial x^{b}}\frac{\partial x^{c'}}{\partial x^{c}} = \Gamma^{a'}_{~b'c'}\frac{\partial x^{a}}{\partial x^{a'}}\frac{\partial x^{b'}}{\partial x^{b}}\frac{\partial x^{c'}}{\partial x^{c}}\label{gamma to gamma prime}\\
K & = K'\label{K to K prime}\\
\end{align}
Plug Eq.~(\ref{alpha to alpha prime}), Eq.~(\ref{beta to beta prime}) and Eq.~(\ref{K to K prime}) into Eq.~(\ref{1 + log slicing}), we have
\begin{align}
\partial_{t}\alpha' - \left(\beta^{c'} - \frac{\partial x^{c'}}{\partial x^{d}}\frac{\partial x^{d}}{\partial t'}\right)\frac{\partial x^{c}}{\partial x^{c'}}\partial_{c}\alpha' & = -2\alpha' K'\notag\\
\partial_{t'}\alpha' + \frac{\partial x^{c'}}{\partial t}\partial_{c'}\alpha' - \beta^{c'}\partial_{c'}\alpha' + \frac{\partial x^{c'}}{\partial x^{d}}\frac{\partial x^{d}}{\partial t'}\partial_{c'}\alpha' & = -2\alpha' K'\notag\\
\partial_{t'}\alpha' + \frac{\partial x^{c'}}{\partial t}\partial_{c'}\alpha' - \beta^{c'}\partial_{c'}\alpha' - \frac{\partial x^{c'}}{\partial t}\frac{\partial t}{\partial t'}\partial_{c'}\alpha' & = -2\alpha' K'\notag\\
\partial_{t'}\alpha'  - \beta^{c'}\partial_{c'}\alpha' & = -2\alpha' K'
\end{align}
Hereby we reached the identical equation as Eq.~(\ref{transformed 1 + log slicing}).

Now let's plug Eqs.~(\ref{beta to beta prime} - \ref{gamma to gamma prime}) into Eq.~(\ref{Gamma-driver shift}) and write the regular Gamma-driver shift condition in terms of the transformed variables. 

Since this is a long equation, let's first examine the left hand side, 
\begin{align}
& \partial_{t}\left[\left(\beta^{a'} - \frac{\partial x^{a'}}{\partial x^{c}}\frac{\partial x^{c}}{\partial t'}\right)\frac{\partial x^{a}}{\partial x^{a'}}\right] - \left[\left(\beta^{b'} - \frac{\partial x^{b'}}{\partial x^{c}}\frac{\partial x^{c}}{\partial t'}\right)\frac{\partial x^{b}}{\partial x^{b'}}\right]\partial_{b}\left[\left(\beta^{a'} - \frac{\partial x^{a'}}{\partial x^{c}}\frac{\partial x^{c}}{\partial t'}\right)\frac{\partial x^{a}}{\partial x^{a'}}\right]\notag\\
= & \partial_{t}\left(\beta^{a'}\frac{\partial x^{a}}{\partial x^{a'}} - \frac{\partial x^{a}}{\partial t'}\right) - \left(\beta^{b'} - \frac{\partial x^{b'}}{\partial x^{c}}\frac{\partial x^{c}}{\partial t'}\right)\partial_{b'}\left(\beta^{a'}\frac{\partial x^{a}}{\partial x^{a'}} - \frac{\partial x^{a}}{\partial t'}\right)\notag\\
= & \left(\partial_{t'}\beta^{a'}\right)\frac{\partial x^{a}}{\partial x^{a'}} - \left(\beta^{b'}\partial_{b'}\beta^{a'}\right)\frac{\partial x^{a}}{\partial x^{a'}} + 2\beta^{a'}\left(\partial_{t'}\frac{\partial x^{a}}{\partial x^{a'}}\right) - \frac{\partial^{2}x^{a}}{\partial t'^{2}}\notag\\
= & \left(\partial_{t'}\beta^{a'} - \beta^{b'}\partial_{b'}\beta^{a'} - 2\beta^{b'}\frac{\partial x^{c}}{\partial x^{b'}}\frac{\partial^{2}x^{a'}}{\partial x^{c}\partial t} + \frac{\partial^{2} x^{a'}}{\partial t^{2}} + 2\frac{\partial^{2} x^{a'}}{\partial t\partial x^{c}}\frac{\partial x^{c}}{\partial t'}\right)\frac{\partial x^{a}}{\partial x^{a'}}
\end{align}
Then the right hand side is
\begin{align}
\frac{3}{4}\sqrt{g'}^{2/3}\left(\Gamma^{a'}_{~b'c'}g^{b'c'} + \frac{1}{3}g^{a'b'}\Gamma^{c'}_{~b'c'}\right)\frac{\partial x^{a}}{\partial x^{a'}} - \eta\left(\beta^{a'} - \frac{\partial x^{a'}}{\partial x^{b}}\frac{\partial x^{b}}{\partial t'}\right)\frac{\partial x^{a}}{\partial x^{a'}}
\end{align}
Multiply $\partial x^{a'}/\partial x^{a}$ on both sides of the equation, we have
\begin{equation}
\partial_{t'}\beta^{a'} - \beta^{b'}\partial_{b'}\beta^{a'} - 2\beta^{b'}\frac{\partial x^{c}}{\partial x^{b'}}\frac{\partial^{2}x^{a'}}{\partial x^{c}\partial t} + \frac{\partial^{2} x^{a'}}{\partial t^{2}} +  2\frac{\partial^{2} x^{a'}}{\partial t\partial x^{c}}\frac{\partial x^{c}}{\partial t'} = \frac{3}{4}{\tilde \Gamma}^{a'} - \eta\left(\beta^{a'} - \frac{\partial x^{a'}}{\partial x^{b}}\frac{\partial x^{b}}{\partial t'}\right)\label{inverted Gamma-driver shift}
\end{equation}
and one can easily check that Eq.~(\ref{inverted Gamma-driver shift}) is equivalent to Eqs.~(\ref{transformed Gamma-driver shift x} - \ref{transformed Gamma-driver shift z}).

The damped-wave gauge condition is easier to derive and check. Due to the transformed results of $\alpha'$, $\beta^{a'}$ and $g'$ we list above, the transformed damped-wave gauge condition is
\begin{subequations}\label{transformed damped gauge}
\begin{align}
H^{a'} & = -\mu_{S}\frac{\beta^{a'} - \frac{\partial x^{a'}}{\partial x^{a}}\frac{\partial x^{a}}{\partial t'}}{\alpha'}\\
H_{\perp'} & = -\mu_{L}\log\left(\frac{\sqrt{g'}}{\alpha'}\right)
\end{align}
\end{subequations}
Using Eqs.~(\ref{beta to beta prime} - \ref{gamma to gamma prime}), we can write the transformed damped-gauge condition from Eq.~(\ref{damped gauge vector}) and Eq.~(\ref{damped gauge scalar}) as well
\begin{subequations}\label{inverted damped gauge}
\begin{align}
H^{a'}\frac{\partial x^{a}}{\partial x^{a'}} & = - \mu_{S}\frac{\left( \beta^{a'} - \frac{\partial x^{a'}}{\partial x^{b}}\frac{\partial x^{b}}{\partial t'}\right)\frac{\partial x^{a}}{\partial x^{a'}}}{\alpha'}\\
H_{\perp'} & = - \mu_{L}\log\left(\frac{\sqrt{g'}}{\alpha'}\right)
\end{align}
\end{subequations}
It is obvious that Eqs.~(\ref{transformed damped gauge}) and Eqs.~(\ref{inverted damped gauge}) are equivalent to each except for an overall factor $\partial x^{a}/\partial x^{a'}$. 

%%%%%%%%%%%%%% needs to revise the summary %%%%%%%%%%%%%%%
In this paper, we have investigated the transformation rules of 3 + 1 variables under foliation preserving coordinate transformation. The non-tensor terms are modified to their covariant counterparts so that we are able to formulate the covariant gauge conditions. We have presented the covariant 1 + log slicing and Gamma-driver shift conditions. In the last part of this paper, a specific coordinate transformation example is given as a guideline for people want to implement these covariant gauge conditions in their numerical work. In the future, we plan to use the results in this paper to achieve a symmetric hyperbolic generalized harmonic formulation for general relativity.  

